\chapter{Measurable functions and integration} \label{chap:meas-func-int}
\section{Measurable functions}\label{sec:measurable-functions}

\begin{defn}
    Given two measurable spaces $(X,\mathcal{M})$ and $(Y,\mathcal{N})$, a function $f\colon X \to Y$ is called a \df{measurable function} if $f^{-1}(A) \in \M$ for all $A \in \mathcal{N}$.
    
    We would stress that the function is $\M/\mathcal{N}$-measurable if the context is not clear. When $(Y,\mathcal{N}) = (\R,\B)$, we usually say $f$ is $\M$-measurable\footnote{Now be aware that either a set or a function may be called $\M$-measurable.}. Therefore when $\M=\B_X$ or $\mathcal{L}_X$, $f$ would be called Borel or Lebesgue measurable, respectively.
\end{defn}

Check on your own that compositions of measurable functions is measurable.

To check measurability, it suffices to just check preimage condition for a collection of subsets that generates the image $\sigma$-algebra $\mathcal{N}$. This is the content of the next proposition, and is a direct consequence of \cref{prop:induce-s-alg-meas-map}\ref{enu:forward-ind-s-alg}.
\begin{prop}\label{prop:measurability-generate}
    If $\mathcal{N}$ is generated by $\mathcal{E}$, then $f\colon X \to Y$ is $\M/\mathcal{N}$-measurable if and only if $f^{-1}(E) \in \M$ for all $E \in \mathcal{E}$.    
\end{prop}

With this sufficient condition in mind, it is easy to check that \begin{itemize}
    \item continuous functions between topological spaces are Borel measurable, and
    \item increasing/decreasing functions from $\R$ to $\R$ are Borel measurable.
\end{itemize}

Given a set $X$, a measurable space $(Y,\mathcal{N})$, and a function $f\colon X \to Y$, then by \cref{prop:induce-s-alg-meas-map}\ref{enu:forward-ind-s-alg} we know \[
    \{f^{-1}(A):A\in \mathcal{N}\}
\] is the smallest $\sigma$-algebra on $X$ that makes $f$ measurable. We call it the \df[sigma-algebra generated by@$\sigma$-algebra generated by!a function]{$\sigma$-algebra generated by $f$}, denoted by $\sigma(f)$.

More generally, consider a collection of measurable spaces $(Y_\alpha,\mathcal{N}_\alpha)$ over all $\alpha \in I$. Suppose we are given $f_\alpha\colon X \to Y_\alpha$ for all $\alpha$. The \df[sigma-algebra generated by@$\sigma$-algebra generated by!functions]{$\sigma$-algebra generated by the class of functions $\{f_\alpha\}_{\alpha \in I}$} on $X$ is defined to be 
\[
    \sigma(\{f_\alpha\}_{\alpha\in I}) = \sigma\bigl(\cup_{\alpha \in I} \{f^{-1}(A_\alpha):A_\alpha\in \mathcal{N}_\alpha\}\bigr).
\] (Recall that union of $\sigma$-algebras is not necessarily a $\sigma$-algebra.)

\begin{prop}
    For any $\sigma(f)/\B(\R)$ measurable function $\phi$, there is a Borel-measurable function $g$ such that $\phi = g \circ f$.
\end{prop}

\begin{namedthm}[Simple function approximation]
    Given $f \in L^+(X,\A)$, there exists a sequence of nonnegative simple functions $\{s_n\}_{n=1}^\infty$ such that $s_n \uparrow f$ pointwise. Furthermore $s_n \to f$ uniformly on any set on which $f$ is bounded.
\end{namedthm}

Note that the ``furthermore'' part essentially means that every nonnegative bounded measurable function is the increasing uniform limit of nonnegative simple functions.

Folland Ex 2.9

Baire $\sigma$-algebra

\section{Nonnegative Lebesgue integrals}

Repartition function is cadlag


\begin{namedthm}[Monotone convergence theorem] \label{thm:MCT}
    If $\{f_n\} \subseteq L^+$ such that $f_n \uparrow f$ , then \[
        \int f = \lim_n \int f_n
    \]
\end{namedthm}
% Note $f_n \uparrow f$ implies $f = \sup_n f_n$ and is thus measurable. Hence $\int f$ makes sense.
\begin{namedthm}[Fatou's lemma] \label{thm:Fatou}
    Let $\{f_n\}\subseteq L^+$, then \[
        \int \bigl(\liminf_n f_n\bigr) \leq \liminf_n \int f_n
    \]
\end{namedthm}

Fatou's lemma is usually useful when one of the two $\liminf$'s is attained.

We see an example when the equality is not achieved. Let the measure space be $(\R,\B,m)$, and set $f_n = n \ind_{(0,1/n]}$. Then $\lim f_n = 0$, while $\liminf \int f_n = 1$.


\section{Signed Lebesgue integrals}
\begin{namedthm}[Lebesgue dominated convergence theorem] \label{thm:DCT}
    If $f_n \to f$ pointwise a.e. [limit], and there exists some nonnegative $g \in L^1$ such that $\abs{f_n} \leq g$ a.e.\ for all $n$, [bound]
    then $f \in L^1$ with the $L^1$ convergence \[
        \lim_n \int \abs{f - f_n} = 0.
    \] (The type of convergence above is known as $L^1$ convergence; see \cref{sec:modes-conv}.) In particular, we have \[
        \int f = \lim_n \int f_n.
    \]
    
\end{namedthm}

\begin{namedthm}[Bounded convergence theorem] \label{thm:bdd-conv}
    When the measure space is finite, it is clear that we can set $g$ in the theorem above to be a nonnegative real number $M$.
\end{namedthm}

% \begin{namedthm}[Bounded convergence theorem] \label{thm:bdd-conv-thm}
%     Say $\mu(X) < \infty$. Let $\{f_n\} \subseteq L^1$. If \begin{enumerate}
%         \item $f_n \to f$ pointwise a.e.,
%         \item and there exists some $M \in \R^+$ such that $\abs{f_n} \leq M$ a.e.\ for all $n$, 
%     \end{enumerate}
%     then $f \in L^1$ with \[
%         \lim_n \int \abs{f - f_n} = 0.
%     \] In particular, we have \[
%         \int f = \lim_n \int f_n.
%     \]
% \end{namedthm}

The above theorems are sometimes used to establish the continuity of functions in the form of integrals.

\begin{namedthm}[Markov's inequality] \label{thm:Markov-ms}
    Let $f\colon X \to \R$ be measurable and $\phi\colon \R \to [0,\infty)$ be increasing (and hence measurable). Then for any $a\in \R$ with $\phi(a)\neq 0$, we have \[
        \mu\{x:f(x) \geq a\} \leq \frac{1}{\phi(a)}\int \phi\circ f\,d\mu.
    \]

    The above statement still holds if we replace all $\R$ above by $[0,\infty)$.
\end{namedthm}
\begin{proof}
    Fix $a$ with $\phi(a)\neq 0$. Using $\phi$ is increasing and nonnegative, we have \begin{align*}
        \phi(a)\mu\{x:f(x) \geq a\} & \leq \int_{\{x:f(x) \geq a\}} \phi(a)\,d\mu(x) \\
        & \leq \int_{\{x:f(x) \geq a\}} \phi\bigl(f(x)\bigr)\,d\mu(x) \\
        & \leq \int \phi\bigl(f(x)\bigr)\,d\mu(x). \qedhere
    \end{align*}
\end{proof}

If we let $\phi(y) = y^p$ ($0<p<\infty$), and use $\abs f$ in place of $f\colon X\to \R$, then we get for any $a > 0$, \begin{equation}
    \mu\{x:\abs{f}\geq a\} \leq \frac{1}{a^p}\int \abs{f}^p\,d\mu. \label{eq:Lp-Markov}
\end{equation}

\begin{namedthm}[Jensen's Inequality] \label{thm:Jensen-ms}
    Let $\mu$ be a probability measure, and $f\in L^1$. Suppose $I$ is an interval containing the range of $f$, and we have a convex function $\phi\colon I\to \R$. % such that $\phi\circ f\in L^1$.
    Then \begin{equation}
        \phi\biggl(\int f\,d\mu\biggr) \leq \int \phi\circ f \,d\mu. \label{eq:jensen}
    \end{equation}

    We do not ask $\phi \circ f\in L^1$. When $\phi \circ f \notin L^1$, the integral attains $+\infty$.
\end{namedthm}

Equality condition

% In particular, if $\phi$ is bounded below, then we can drop the integrability assumption on $\phi \circ f$. If \[\int \abs{\phi \circ f}= \int_{\{\phi \circ f \geq 0\}} \phi\circ f + \int_{\{\phi \circ f < 0\}} -\phi\circ f= \infty\] and $\phi$ is bounded below, then \[\int \phi \circ f\,d\mu = \int_{\{\phi \circ f \geq 0\}} \phi\circ f - \int_{\{\phi \circ f < 0\}} -\phi\circ f = \infty.\] Hence the inequality \eqref{eq:jensen} trivially holds.

\section{Connections to the Riemann theory}

\begin{namedthm}[Bounded convergence theorem (Riemann integration)]
    
\end{namedthm}

We use $\int_a^b f(x)\,dx$ for Riemann integrals, and $\int_{[a,b]} f(x)\,dm(x)$ for Lebesgue integrals.

Improper Riemann integral 

An improper Riemann integral is Lebesgue integrable if it is absolutely convergent.

 but $\frac{\sin x}{x} \ind_{[0,\infty)}$ is not Lebesgue integrable.

\begin{namedthm}[Dirichlet integral]
    Let us show $\int_{0}^\infty \frac{\sin x}{x}\,dx = \pi/2$. The easiest solution is to use the double integral trick.
\end{namedthm}

\section{Modes of convergence} \label{sec:modes-conv}
\begin{defn}
    For a sequence of measurable functions ${f_n}$, we say $f_n$ converges to some function $f$ 
    \begin{itemize}
        \item \df[convergence!almost everywhere]{almost everywhere} (a.e.) if \[
            \mu\{x: \lim_n f_n(x) = f(x)\}^\cpl = 0.
        \]
        \item \df[convergence!in L-p@in $L^p$]{in $L^p$} ($1 \leq p <\infty$), if $\int \abs{f_n}^p <\infty$ for all $n$, and \[
            \int \abs{f_n - f}^p \to 0.
        \]
        In \cref{sec:Lp-1-infty} we will show that the limiting function $f$ also has $\int \abs{f}^p < \infty$, along with other basic facts about $L^p$ spaces.
        \item \df[convergence!in measure]{in measure} if for any $\epsilon > 0$, \begin{equation} \label{eq:in-measure}
            \lim_n \mu\{x : \abs{f_n(x) - f(x)} > \epsilon\} = 0.
        \end{equation}
    \end{itemize}
    We say $\{f_n\}$ is 
    \begin{itemize}
        \item \df{Cauchy/fundamental in measure} if for any $\epsilon > 0$, there exists $N \in \N$ such that for all $m > n \geq N$, \begin{equation} \label{eq:Cauchy-in-measure}
            \mu\{x:\abs{f_n(x) - f_m(x)} > \epsilon \} < \epsilon
        \end{equation}
    \end{itemize}
    Note that the ``$>$'' in both \eqref{eq:in-measure} and \eqref{eq:Cauchy-in-measure} can be replaced by ``$\geq$'', obviously. It suffices to use only one $\epsilon$ in \eqref{eq:Cauchy-in-measure} because we can always choose the smaller of two distinct $\epsilon$'s.
\end{defn}

% We will only discuss $L^1$ convergence in this chapter because the basic facts about $L^p$ spaces are reserved for \cref{sec:Lp-1-infty}. These basic facts will generalize most results about $L^1$ convergence to $L^p$ convergence.

\begin{thm}[(relationships between different modes of convergence)] \label{thm:relation-modes-conv} \leavevmode
    \begin{enumerate}
        \item The a.e.-limit, $L^p$-limit, and limit-in-measure are all unique a.e.
        \item $f_n \to f$ in measure implies $\{f_n\}$ is Cauchy in measure; and $\{f_n\}$ being Cauchy in measure implies $f_n \to f$ in measure for some $f$.
        \item \label{enu:measure-subseq-ae} $f_n \to f$ in measure implies there exists a subsequence $\{f_{n_k}\}$ that converges a.e.\ to $f$ as $k \to \infty$.
        \item Convergence in $L^p$ implies convergence in measure.
        \item \label{enu:ae-implies-meas} If the measure space is finite, then convergence a.e.\ implies convergence in measure. (Hence in a finite measure space, if a function converges a.s./in measure and in $L^p$, then the two limits should agree.)
        \item $f_n \to f$ in measure if and only if for every subsequence $f_{n_k}$ there exists a further subsequence $f_{n_{k_j}}$ that converges in measure to $f$. % In particular when the measure space is finite, if every subsequence $f_{n_k}$ there exists a further subsequence $f_{n_{k_j}}$ that converges a.e.\ to $f$, then $f_n \to f$ in measure.
    \end{enumerate}
\end{thm}
\begin{proof} \leavevmode
    \begin{enumerate} 
        \item The first is obvious. The second follows from \nameref{thm:Minkowski-ineq}; in particular when $p=1$ we may just use the triangular inequality.

        For the third one, suppose $f$ and $g$ are both limits-in-measure. Then for any $\epsilon > 0$, it holds that \[
            \lim_n \mu\bigl\{x : \abs{f_n(x) - f(x)} > \epsilon/2 \text{ or }\abs{f_n(x) - g(x)} > \epsilon/2\bigr\} = 0.
        \] This implies \[
            \mu\{x : \abs{f(x) - g(x)} > \epsilon\} = 0.
        \] The result follows by $\epsilon$ being arbitrary.

        We emphasize that the containment relation \begin{equation} \label{eq:in-measure-containment}
            \abs{f(x) - g(x)} > \epsilon \implies \abs{f(x) - h(x)} > \epsilon/2 \text{ or }\abs{h(x) - g(x)} > \epsilon/2
        \end{equation}
        for some appropriate functions $f,g,h$, is the common trick used to prove convergence in measure.
        \item The first claim is easy and left to the readers, again by the containment relation \eqref{eq:in-measure-containment}. For the second one, the idea is to construct a subsequence that converges pointwise a.e.\ to some function, which we prove is our $f$.

        For each $k \in \N$, define $g_k = f_{n_k}$, where $n_k$ is the smallest integer such that \begin{equation} \label{eq:construct-subseq}
             \mu\{x:\abs{f_n(x) - f_m(x)} > 2^{-k}\} < 2^{-k}\quad \text{for all }m\geq n \geq n_k.
        \end{equation}
        We claim this appropriately picked sequence $g_k=f_{n_k}$ converges for a.e.\ $x$. This is equivalent to proving that $g_k$ is a.e.\ Cauchy.

        Note that $g_k$ is exactly the desired subsequence in part~\ref{enu:measure-subseq-ae}, by our claim that convergence in measure implies Cauchy in measure.
        
        Define \[E_j = \{x:\abs{g_j(x) - g_{j+1}(x)}\geq 2^{-j}\}.\] This gives \[
            \mu\biggl(\bigcup_{j=k}^\infty E_j\biggr) \leq \sum_{j=k}^\infty 2^{-j} = 2^{-k+1},
        \] which goes to $0$ as $k \to \infty$. Hence $\mu(\limsup_k E_k) = 0$, that is, a.e.\ $x$ falls in $\{E_k\}_{k=1}^\infty$ eventually.\footnote{The reader might notice that we have implicitly proved and used \nameref{thm:BorelCantelli-meas-th} here. This is how convergence a.e.\ is usually proved, and we will see more applications of this when discussing probability. The main reason we have not invoked Borel--Cantelli directly is that we will use the inequality again in the next section of the proof.}
        To be precise, there is this $N\in \N$ such that for all $k \geq N$, for all $m > n \geq k$, it holds for a.e.\ $x$ that \begin{align*}\begin{split}
            \abs{g_n(x) - g_m(x)} & \leq \sum_{j=n}^{m-1}\abs{g_j(x) - g_{j+1}(x)} \\
            & \leq 2^{-n+1} \leq 2^{-k+1}. \end{split}
        \end{align*}

        Hence we have a pointwise a.e.\ limit $f$ of $\{g_k\} = \{f_{n_k}\}$. In fact $g_k$ converges in measure to $f$ as well. (If the measure space is finite we may use part~\ref{enu:ae-implies-meas}, but this is true in general.) 

        Fix $k$, we have proved already that $\mu(\bigcup_{j=k}^\infty E_j)\leq 2^{-k+1}$; and for $x \notin \bigcup_{j=k}^\infty E_j$, for $m > n \geq k$, \[
            \abs{g_n(x) - g_m(x)} \leq 2^{-k+1}.
        \]
        Take $m \to \infty$ in the inequality above, and we have for $x \notin \bigcup_{j=k}^\infty E_j$, there is $k$ such that for all $n \geq k$, \[
            \abs{g_n(x) - f(x)} \leq 2^{-k+1},
        \] This yields $g_k \to f$ in measure.

        The final step is to use this to show $f_n \to f$ in measure. We again resort to the containment relation \eqref{eq:in-measure-containment}: \[
            \abs{f_n(x) - f(x)} > \epsilon \implies 
            \underbrace{\abs{f_n(x) - g_k(x)} >\epsilon/2}_{\text{terms in a Cauchy sequence}} \text{ or }\underbrace{\abs{g_k(x) - f(x)} > \epsilon/2}_{\substack{\text{terms in a sequence} \\ \text{that converges in measure}}}.
        \] Hence $f_n \to f$ in measure, as desired.
        \item Contained in the previous part.
        \item This is clearly a consequence of \eqref{eq:Lp-Markov}.
        \item Fix $\epsilon > 0$, define $E_n = \{x:\abs{f_n(x) - f(x)} < \epsilon\}$. Recall $\liminf_n E_n$ consists of all $x$ such that $\abs{f_n(x) - f(x)} < \epsilon$ eventually.

        Since $\epsilon$ has been fixed, we have $\liminf_n E_n$ should contain all $x$ such that $f_n(x) \to f(x)$. By assumption \[\mu(X) = \mu\{x:f_n \to f\} \leq \mu\bigl(\liminf_n E_n\bigr) \leq \liminf_n \mu(E_n),\] which now implies $\mu(X) = \liminf_n \mu(E_n) = \lim_n \mu(E_n)$. This exactly means $f_n \to f$ in measure.
        \item The ``only if'' direction is trivial. The ``if'' direction, on the other hand, clearly resembles \cref{prop:subseq-further-subseq-top-space}: fix $\epsilon > 0$ and consider $y_n = \mu\{x:\abs{f_n(x) - f(x)} > \epsilon\}$. \qedhere
     \end{enumerate}
\end{proof}

\begin{exa}
    Part~\ref{enu:ae-implies-meas} is not true in general for infinite measure spaces: let $\mu$ be Lebesgue measure on $\R$, the sequence of functions specified by $f_n = \ind_{[n,n+1)}$ converges to $0$ a.e., but not in measure.

    Convergence in $L^p$ (and hence in measure) does not imply convergence a.e.: specify $f_n =\ind_{[j/2^k,(j+1)/2^k)}$, where $n = 2^k+j$ with $0\leq j < 2^k$. The sequence dyadically moves across $[0,1)$, in the sense that $f_1 = \ind_{[0,1)}$, $f_2 = \ind_{[0,1/2)}$, $f_3 = \ind_{[1/2,1)}$, $f_4 = \ind_{[0,1/4)}$, $f_5 = \ind_{[1/4,1/2)}$, and so on. The sequence converges to $0$ in $L^1$, but not a.e. This is a very important example to remember.

    Pointwise, a.e., and uniform convergence does not give $L^p$ convergence: consider $f_n = \frac{1}{n} \ind_{[n,n+1)}$, $n\ind_{[0,1/n)}$, and $\frac{1}{n} \ind_{[0,n)}$ respectively, which converges pointwise, a.e., and uniformly to $0$ but not in $L^1$.
\end{exa}

\begin{xca}
    Give a proof of \cref{thm:relation-modes-conv}\ref{enu:ae-implies-meas} using the \flcnameref{thm:bdd-conv}.
\end{xca}

\begin{fact}
    Convergence a.e.\ is preserved under continuous composition: given $f_n \to f$ a.e. and a continuous function $\Psi\colon \R \to \R$, then $\Psi(f_n)\to \Psi(f)$ a.e.
\end{fact}

\begin{cor}
    Let $\mu$ be finite, and $f_n \to f$ and $g_n \to g$ in measure. Say $\Psi\colon \R^2 \to \R$ is a continuous function, then $\Psi(f_n,g_n) \to \Psi(f,g)$ in measure. In particular, $f_n + g_n \to f + g$ and $f_n g_n \to fg$ in measure.
\end{cor}
\begin{proof}
    The measurabilities of $\Psi(f_n,g_n)$ and $\Psi(f,g)$ are left to the readers. Suppose by contradiction that $\Psi(f_n,g_n) \not\to \Psi(f,g)$ in measure, then for some $\epsilon > 0$ and a subsequence $\{(f_{n_k},g_{n_k})\}_k$ of $\{(f_n,g_n)\}_n$ we have \begin{equation}
        \mu\bigl\{x: \bigl\vert\Psi\bigl(f_{n_k}(x),g_{n_k}(x)\bigr) - \Psi\bigl(f(x),g(x)\bigr)\bigr\vert > \epsilon \bigr\} \geq\epsilon. \label{eq:subseq-never-conv-in-meas}
    \end{equation} Recall the construction of the subsequence in \cref{thm:relation-modes-conv}\ref{enu:measure-subseq-ae}. An obvious modification of $n_k$ there, or $n_{k_j}$ in our context, gives us a subsequence $\{n_{k_j}\}$ of $\{n_k\}$ such that simultaneously \[
        f_{n_{k_j}} \to f \quad \text{and} \quad g_{n_{k_j}} \to g\quad\text{a.e.}
    \] It follows that \[
        \Psi\bigl(f_{n_{k_j}}(x),g_{n_{k_j}}(x)\bigr) \to  \Psi\bigl(f(x),g(x)\bigr) \quad\text{a.e.},
    \] and hence in measure. But this contradicts our pick of $\{n_k\}$ specified by \eqref{eq:subseq-never-conv-in-meas}.
\end{proof}

This proof shows the power of both part~\ref{enu:measure-subseq-ae} and \ref{enu:ae-implies-meas}. Remember that extracting an a.e.\ convergent can be helpful in many proofs involving convergence in measure.

\begin{rem}
    One can prove directly that two most important cases, $f_n + g_n \to f+g$ and $f_ng_n \to fg$ in measure above, without using proof by contradiction. One will also see that it is unnecessary to assume finite measure space when proving $f_n + g_n \to f + g$ in measure. We leave these as an exercise to the interested readers.
\end{rem}

\begin{xca}
Use \cref{thm:relation-modes-conv}\ref{enu:measure-subseq-ae} to prove the \flcnameref{thm:MCT} and \nameref{thm:Fatou} with convergence in measure.
\end{xca}

\section{Littlewood's second and third principles} \label{sec:Littlewood-2nd-3rd}
\begin{namedthm}[Egoroff's theorem] \label{thm:Egoroff}
    Say $\mu(X)<\infty$. Let ${f_n}$ be a sequence of $\A$-measurable functions from $X$ to $\R$ (or $\C$). Then for all $\epsilon > 0$, there exists some measurable set $E$ such that \[
        \mu(E^\cpl) < \epsilon, \quad \text{while } f_n \to f \text{ uniformly on } E.
    \]
    We call this conclusion $f_n$ converges to $f$ \df[convergence!almost uniformly]{almost uniformly}.
\end{namedthm}

% \begin{xca}[\cite{[}{]}{folland1999}]
%     If $f_n \to f$ almost uniformly, then $f_n \to f$ a.e.\ and in measure.
% \end{xca}
We mention that it is a good exercise to prove the \flcnameref{thm:bdd-conv} using this result.
\begin{namedthm}[Luzin's theorem]
    Let $f\colon [a,b] \to \R$ (or $\C$) be a Borel measurable function. Then for every $\epsilon >0$, there exists a closed set $F \subseteq [a,b]$ such that $f|_F$ is continuous while $m([a,b] - F) < \epsilon$.
\end{namedthm}

\section{Uniformly integrable functions}

Use the material we have discussed so far to prove the following result.
\begin{xca}[\cite{Royden_2023}] \label{xca:abs-cont-int-motiv}
Let $f \in L^1(\mu)$. Then
\begin{enumerate}
    \item \label{enu:abs-cont-int}for all $\epsilon > 0$, there is a $\delta > 0$ such that \[
    \mu(E) < \delta \implies \int_E \abs{f} \,d\mu < \epsilon;
\]
    \item moreover, for each $\epsilon > 0$, there is some $X_0$ with $\mu(X_0) < \infty$ such that \[
    \int_{X - X_0} \abs{f} < \epsilon.
\]
\end{enumerate}
\end{xca}

Notice that \[
    \biggl\vert\int_E f \,d\mu \biggr\vert \leq \int_E \abs{f} \,d\mu = \biggl\vert\int_{E \cap \{f \geq 0\}} f\,d\mu\biggr\vert + \biggl\vert\int_{E \cap \{f < 0\}} -f \,d\mu\biggr\vert.\]
Hence conclusion~\ref{enu:abs-cont-int} is equivalent to $\forall\,\epsilon > 0$, $\exists\, \delta > 0$ such that \[
    \mu(E) < \delta \implies \biggl\vert\int_E f \,d\mu\biggr\vert <\epsilon.
\]

This motivates the next definition, which requires \ref{enu:abs-cont-int} to hold uniformly for a class of integrable functions.

\begin{defn}
    A set of functions $\F \subseteq L^1(\mu)$ has \df{uniformly absolutely continuous integrals} if for every $\epsilon > 0$, there exists $\delta > 0$ such that \[
        \mu(E) < \delta \implies \int_E \abs{f} \,d\mu <\epsilon \text{ for all $f \in \F$},
    \] or equivalently, \[ \biggl\vert\int_E f \,d\mu\biggr\vert <\epsilon \text{ for all $f \in \F$}.
    \]
\end{defn}

The term ``absolutely continuous'' that appear in the definition above is related the notion of an absolutely continuous pair of measures we will discuss in \cref{sec:signed}. Since for $f \in L^1(X,\A,\mu)$, $\nu(E) = \int_E \abs{f} \,d\mu$ defines a finite positive measure $\nu$ on $\A$ that is absolutely continuous with respect to $\mu$. This immediately proves conclusion~\ref{enu:abs-cont-int} in \cref{xca:abs-cont-int-motiv}.

\begin{defn}
    A set of functions $\F\subseteq L^1(\mu)$ is \df{uniformly integrable} if \[
        \lim_{C \to \infty} \sup_{f\in \F} \int_{\{\abs{f} > C\}} \abs{f}\,d\mu = 0.
    \]
\end{defn}

These two definitions are quite obviously related, as stated by the next proposition.

\begin{prop}
    Let $\mu$ be finite, then $\F$ is uniformly integrable if and only if it is bounded in $L^1$ and also has uniformly absolutely continuous integrals.
\end{prop}

\begin{fact}
    Any finite collection of $L^1$ functions is uniformly integrable. Any collection of bounded functions is uniformly integrable.
\end{fact}

The following proposition gives an easy sufficient condition for uniform integrability. Note that this $p > 1$ will come back later 

\begin{prop}
    Suppose there exists some $p > 1$ such that the collection $\F$ of functions is $L^p$ bounded (i.e., $\sup_{f\in \F} \int \abs{f}^p \,d\mu <\infty$) then the collection $\F$ is uniformly integrable.
\end{prop}
\begin{proof}
    This might as well be left as an exercise, but we write out the proof due to its importance.

    Let $C > 0$, we first observe that \[
        \int_{\{\abs{f} > C\}} \abs{f}^p \geq C^{p-1} \int_{\{\abs{f} > C\}} \abs{f}.
    \] Hence \[
        0 \leq \sup_{f} \int_{\{\abs{f} > C\}} \abs{f} \leq \frac{1}{C^{p-1}}\sup_f\int_{\{\abs{f} > C\}} \abs{f}^p.
    \] Now with the assumption and $p > 1$, by the squeeze theorem we conclude that the collection $\F$ is uniformly integrable.
\end{proof}

\begin{namedthm}[Vitali convergence theorem]
    Suppose $\mu$ is finite. Let $\{f_n\} \subseteq L^1(X,\A,\mu)$, then the following are equivalent: 
    \begin{enumerate}
        \item $f \in L^1$ with $f_n \to f$ in $L^1$.
        \item $f_n \to f$ in measure, and $\{f_n\}$ is uniformly integrable.
    \end{enumerate}
\end{namedthm}
\section{Continuity and differentiability of parametrized functions}

\section{Image measures} \label{sec:image-measure}
Consider a measure space $(X,\mathcal{M},\mu)$ and a measurable space $(Y,\mathcal{N})$. If we have an $(\mathcal{M},\mathcal{N})$-measurable function $\phi\colon X \to Y$, then we can define a function $\mu_{*}\colon \mathcal{N} \to [0,\infty]$ given by \[
    \mu_{*}(E) =  \mu(\phi^{-1}E)
\] for all $E\in \mathcal{N}$. This turns out to a measure on $(Y,\mathcal N)$, and we call this the \df{image/pushforward measure} of $\mu$ by $\phi$, denoted by $\phi_*\mu$ or $\phi_{\#}\mu$.

Image measure characterizes change of variables, which is of basic importance in mathematics. We will use image measures later in \cref{sec:cov,sec:polar,sec:moment-indep-joint}.

We state the main result below.
\begin{prop} \label{prop:image-meas-cov}
    Under the conditions stated above, let $g\in L^+(Y,\mathcal{N})$ or $g \circ \phi \in L^1(X,\mathcal{M},\mu)$. Then \begin{equation*}
        \int_X g\bigl(\phi(x)\bigr) \,d\mu(x) = \int_Y g(y) \,d\mu_*(y). %\label{eq:image-m}
    \end{equation*}
\end{prop}
\begin{proof}
    When $g = \ind_E$ for $E \in \mathcal{N}$, we have \[
        \text{LHS} = \mu\{x : \phi(x) \in E\} = \mu(\phi^{-1} E) \quad \text{and} \quad 
        \text{RHS} = \mu_{*}(E).
    \] Now extend this to simple functions, then nonnegative functions, and then integrable functions.
\end{proof}