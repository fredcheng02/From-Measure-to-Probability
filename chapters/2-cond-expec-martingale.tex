\chapter{Conditional expectations and discrete martingales}
\section{Conditional expectations}
\begin{defn} \label{def:cond-expec}
    Let $\E \abs{X} < \infty$, and $\G$ be a sub-$\sigma$-field of $\F$. Define the \df[conditional expectation!for $L^1$ random variables]{conditional expectation} of $X$ given $\G$ to be the random variable $Y$ satisfying \begin{enumerate}
        \item $Y$ is $\G$-measurable; 
        \item \label{enu:equal-int-cond-expec} $\E(Y \ind_{G}) = \E(X \ind_G)$ for all $G \in \G$.
    \end{enumerate}
    This $Y$ is denoted by $\E(X \giv \G)$.
\end{defn}

We first show that the above definition makes sense from a purely measure-theoretic point of view, and is unique a.s.
Notice that the function $\nu\colon \mathcal{G} \to \R$ given by \begin{equation} \label{eq:cond-expec-signed-meas}
    \nu(G) = \E(X\ind_G) = \int_G X \,dP
\end{equation} is a signed measure, and $\nu \ll P|_\G$. Therefore by the \nameref{thm:Radon-Nikodym} for a signed measure and a finite positive measure, there exists a random variable $Y$, unique in $L^1(\Omega,\G,P|_\G)$, such that \[
    \nu(G) = \int_G Y \,dP = \E(Y\ind_G)
\] for all $G\in \G$. \emph{Be aware that conditional expectations are unique up to measure zero.}

% Note that there does not exist a version of conditional expectation $\E(X\giv \G)$ defined specifically for nonnegative $X$. This is because the $\nu$ we defined in \eqref{eq:cond-expec-signed-meas} may become possibly infinite. We have discussed previously that there does not exist a nice Radon--Nikodym theorem for a general infinite measure $\nu$.

\begin{defn}
    Define the \df{conditional probability} of $A \in \F$ given a sub-$\sigma$-field $\G$ of $\F$ to be $\E(\ind_A \giv \G)$, which we denote by $P(A\giv \G)$.
\end{defn}

% When $X = \ind_A$, the $\nu(G)$ in \eqref{eq:cond-expec-signed-meas} now becomes $P(A\cap G)$. In a first course in probability we define \[
%     P(A \giv G) = \frac{P(A \cap G)}{P(G)}.
% \] % The conditional probability $P(A \giv \G)$ is therefore the Radon--Nikodym derivative of the probability measure $\nu$ (that takes the information $\G$) with respect to the original probability measure $P$.

Our new definitions of conditional expectation and conditional probability are very abstract, and particularly distinct from the undergraduate version, and the following example is almost included in all textbooks, which explains how our new definitions generalizes the old definitions.

\begin{exa}
    Let $\Omega_1,\Omega_2,\dotsc$ be a countable partition of the sample space $\Omega$, where each $\Omega_n$ has strictly positive measure. In an undergraduate class we would define \[
        \E(X\giv \Omega_n) = \frac{\E(X;\Omega_n)}{P(\Omega_n)}
    \] for any $n$. Now define $\G = \sigma(\{\Omega_n\}_{n=1}^\infty)$. It is easy to see that \begin{equation} \label{eq:agree-two-cond-expec}
        \int_{\Omega_n} \frac{\E(X;\Omega_n)}{P(\Omega_n)} \,dP = \int_{\Omega_n} X\,dP.
    \end{equation} We claim that $\E(X\giv \G)$ is given by \[
        Y = \frac{\E(X;\Omega_n)}{P(\Omega_n)} \text{ on each }\Omega_n,
    \] and hence coincides with our undergraduate definition.

    First the candidate $Y$ is $\G$-measurable since it is a constant on each $\Omega_n$. Also since $\{\Omega_n\}$ is a partition of $\Omega$ and generates $\G$, equation~\eqref{eq:agree-two-cond-expec} immediately implies that \[
        \int_G Y\,dP = \int_G X\,dP
    \] for all $G\in \mathcal{G}$. This finishes the proof.

    Now we look at condition probability. Set $X = \ind_A$, and we have \begin{align*}
        P(A \giv \G) & = \E(\ind_A\giv\G) \\ & = \frac{\E(\ind_A\ind_{\Omega_n})}{P(\Omega_n)} \text{ on each }\Omega_n \\ & = \frac{P(A\cap \Omega_n)}{P(\Omega_n)} \text{ on each }\Omega_n,
    \end{align*} which was our undergraduate definition of conditional probability $P(A\giv \Omega_n)$.
\end{exa}

% The Radon--Nikodym derivative should match on partitions

\begin{fact}[(characteristic property)] \label{fact:char-property-CE}
    Let all $X \in \F$ and $Z \in \G$ satisfying $\E \abs{X} < \infty$ and $\E\abs{X Z} < \infty$, we have \[
        \E(\E(X \giv \G) Z) = \E(X Z).
    \] This property characterizes the conditional expectation $\E(X \giv \G)$.  % (The boundedness is required for the expectation to exist automatically. One can make some modifications in certain cases.)
\end{fact}
\begin{proof}
    Left as an exercise, using the standard limiting argument.
\end{proof}

\begin{prop} Let $X,Y \in L^1(\Omega,\F,P)$.
    \begin{enumerate}
        \item For $X$ that is $\G$-measurable, $\E(X \giv \G) = X$.
        \item For $X$ and $\G$ that are independent, $\E(X \giv \G) = \E X$.
        \item Linearity: $\E(aX + Y \giv \G) = a\E(X \giv \G) + \E(Y\giv \G)$.
        \item Monotonicity: if $X \geq Y$ a.s., then $\E(X \giv \G) \geq \E(Y \giv \G)$.
        \item Contractivity (in $L^1$): $\bigl\vert \E(X \giv \G)\bigr\vert \leq \E\bigl(\abs{X} \bigm\vert \G\bigr)$, and taking expectation on both sides gives $\E\bigl(\bigl\vert\E(X \giv \G)\bigr\vert\bigr) \leq \E \abs{X}.$
    \end{enumerate}
\end{prop}

\begin{namedthm}[Conditional Jensen's inequality] \label{thm:cond-Jensen}
    Let $\phi\colon \R \to \R$ be convex, and $X$ and $\phi(X)$ be both integrable, then \[
        \phi\bigl(\E (X\giv \G)\bigr) \leq \E (\phi(X) \giv \G).
    \]
\end{namedthm}

\begin{cor}[(Contraction property)] \label{cor:CE-contraction}
    The conditional expectation $\E(\blank \giv \G)$ is a $1$-Lipschitz linear operator on any $L^p$ ($1 \leq p < \infty$): for $X \in L^p(\Omega,\F,P)$, $\bigl\vert \E(X^p \giv \G)\bigr\vert \leq \E\bigl(\abs{X}^p \bigm\vert \G\bigr)$, and taking expectations on both sides gives
        \[
            \E\bigl(\bigl\vert\E(X \giv \G)\bigr\vert^p\bigr) \leq \E \abs{X}^p.
        \]

    In particular, this implies that if $X_n \to X$ in $L^p$, then $\E(X_n \giv \G) \to \E(X\giv \G)$ in $L^p$.
\end{cor}

\begin{thm}[(alternative Hilbert space definition)]
    Let $X \in L^2(\F)$, which is a Hilbert space. Then $\E(X \giv \G)$ is exactly the projection to the closed subspace $L^2(\G)$. Furthermore, this projection linear operator $\pi\colon L^2(\F) \to L^2(\G)$ can be uniquely extended to a bounded linear operator $\Pi\colon L^1(\F)\to L^1(\G)$, which is exactly the conditional expectation defined by Radon--Nikodym in \cref{def:cond-expec}.
\end{thm}
\begin{proof}
    The \flcnameref{thm:proj-Hilbert} says that it suffices to show that for all $Y \in L^2(\G)$, \[
        \E(\E(X \giv \G) Y) = \E(XY).
    \]
    This is true by \cref{fact:char-property-CE} and $\E(X \giv \G) \in L^2(\G)$, which follows from \cref{cor:CE-contraction}.
    
    To extend the linear operator $\pi$ to a larger domain $L^1(\F)$, recall that $L^2(\F)$ is dense when considered as a metric subspace of $L^1(\F)$, and $L^1(\G)$ is complete. Now consider $\pi$ as a function from $(L^2(\F),\nm{\blank}_1)$ to $(L^1(\G),\nm{\blank}_1)$. We claim this $\pi$ is bounded, in particular $1$-Lipschitz. To see this, it suffices to verify that \[
        \E \bigl\vert\E(X\giv \G)\bigr\vert \leq \E \abs{X}
    \] for all $X \in L^2(\F)$. Now let $A = {\E(X \giv \G) \geq 0}$, then \begin{align*}
        \E \bigl\vert\E(X\giv \G)\bigr\vert & = \E\bigl(\E (X \giv \G)\ind_A\bigr) - \E\bigl(\E(X \giv \G)\ind_{A^\cpl}\bigr) \\
        & = \E(X \ind_A) - \E(X\ind_{A^\cpl}) \leq \E\abs{X}.
    \end{align*}

    With all these information, by \cref{thm:ext-unif-cont-func} we have a continuous linear operator $\Pi\colon L^1(\F) \to L^1(\G)$, and by the uniqueness of the extension, $\Pi$ should exactly be the conditional expectation $\E(\blank \giv \G)$ defined previously.
\end{proof}

$L^2$-contractivity follows from Hilbert subspace projection reduces norm

\begin{namedthm}[Tower property] \label{thm:tower}
    For $\G_1 \subseteq \G_2$, we have \[
        \E(\E(X\giv \G_1) \giv \G_2) = \E(X\giv \G_1) = \E(\E(X\giv \G_2) \giv \G_1).
    \] This means that the iterated conditioning is ultimately conditioning on the smallest $\sigma$-field. Note in particular, we have \[
        \E(\E(X \giv \G)) = \E X.
    \]
\end{namedthm}

\begin{prop}
    For $Y \in \G$, we have $\E(XY \giv \G) = Y\E(X \giv \G)$.
\end{prop}

\begin{prop}
    For two sub-$\sigma$-fields $\G_1$ and $\G_2$ of $\F$, then the following three are equivalent:
    \begin{enumerate}
        \item $\G_1$ and $\G_2$ are independent;
        \item \label{enu:indep-sub-sigma-fields} $\E(X\giv \G_1) = \E X$ for every $X \in L^+(\G_2)$ or $L^1(\G_2)$;
        \item $\E(\ind_{G_2} \giv \mathcal G_1) = P(G_2)$ for every $G_2 \in \G_2$.
    \end{enumerate}
\end{prop}

In particular, let $X$ and $Y$ be two random variables. Consider $\G_1 = \sigma(X)$ and $\G_2 = \sigma(Y)$. Then $X$ and $Y$ are independent if and only if \[
    \E[f(X) \giv Y] = \E f(X)
\] for all $f$ such that $\E \abs{f(X)} < \infty$.

\begin{prop}
    Let $X\colon (\Omega,\F) \to (T,\mathcal T)$ and $Y\colon (\Omega ,\F) \to (S,\mathcal S)$, and say $\G$ is a sub-$\sigma$-field of $\F$. If $X$ is $\G$-measurable and $Y$ is independent of $\G$, then for any $f\colon (T\times S ,\mathcal T\otimes \mathcal S) \to (\R,\B)$ such that $\E\abs{f(X,Y)} < \infty$, we have \[
        \E[f(X,Y)\giv \G] = h(X),\text{ where }h(x) = \E f(x,Y).
    \]
    
    In particular, when $X$ and $Y$ are independent, we have \[
        \E[f(X,Y)\giv X] = h(X).
    \]
\end{prop}

\begin{defn}
    Let $X$ be nonnegative $\F$-measurable, then we define its \df[conditional expectation!for nonnegative random variables]{conditional expectation} given $\G$ to be \[
        \E(X \giv \G) = \lim_{n \to \infty} \E(X \bmin n \giv \G).
    \]
\end{defn}

\nameref{thm:Radon-Nikodym} fails to help us 




\begin{namedthm}[de~Finetti's theorem]
    For a sequence of exchangeable random variables, conditioning on the exchangeabale $\sigma$-field $\mathcal E$, $X_1,X_2,\dotsc$ are i.i.d.\ More precisely, we can show that for any bounded measurable functions $f_j$'s, it holds that \[
        \E\biggl(\prod_{j=1}^n f(X_j)\biggm\vert \mathcal E\biggr) = \prod_{j=1}^n \E[f(X_j) \giv \mathcal E].
    \]
\end{namedthm}

This result roughly says that the conditional distribution of $X_j \giv \mathcal E$ becomes i.i.d. We remind that when $X_j$'s take value in standard Borel spaces, then there is a regular conditional distribution for $X_j \giv \mathcal E$.

\section{Conditional distributions and transition kernels}

\begin{defn} \label{def:reg-cond-dist}
Let $(\Omega,\F)$ and $(S,\mathcal S)$ be two measurable space. A \df{random probability measure} is $\nu\colon \Omega\times \mathcal S \to [0,1]$ such that \begin{enumerate}
    \item for ($P$-a.e.)\ $\omega \in \Omega$, the function $\nu(\omega,\blank)$ is a probability measure on $(S,\mathcal S)$;
    \item for each $A \in \mathcal S$, the function $\omega \mapsto \nu(\omega,A)$ is $\F$-measurable.
\end{enumerate}

Let $Y\colon (\Omega,\F)\to (S,\mathcal S)$, and $\G$ be a sub-$\sigma$-field of $\F$. A \df{regular conditional distribution} of $Y$ given $\G$ is a random probability measure $\nu\colon \Omega\times \mathcal S \to [0,1]$ such that satisfies: 
    \begin{enumerate}[resume]
        \item\label{enu:reg-cond-meas} for each $A \in \mathcal S$, the function $\omega \mapsto \nu(\omega,A)$ is a version of $P(Y\in A\giv \G)$.
    \end{enumerate}
\end{defn}

Recall that the function $P(Y \in A \giv \G)$ is unique only $P$-a.e., hence a version of $P(Y \in A \giv \G)$ means a function defined at each $\omega \in \Omega$. Note that condition~\ref{enu:reg-cond-meas} may be replaced by the following: for each $f$ such that $\E \abs{f(Y)} < \infty$, we have for $P$-a.e.\ $\omega$, \[
    \E [f(Y ) \giv \G] = \int f(y)\,\nu(\omega,dy).
\] This follows by observing that \[\E\bigl(\ind_{\{Y \in A\}} \bigm\vert \G\bigr) (\omega) = P(Y \in A \giv \G) (\omega) =\int \ind_A(y)\,\nu(\omega,dy),\] and then employing the standard approximation argument.

Most often we take $\G = \sigma(X)$, and then our $\nu$ defined above is the \emph{regular conditional distribution} of $Y$ given $X$. In this case however, our notation above turns out to be awkward, since we want to make the role of $\omega$ implicit, but the role of $x = X(\omega)$ explicit. To accomplish this the following general definition is introduced.

\begin{defn}
    Given two measurable spaces $(T,\mathcal T)$ and $(S,\mathcal S)$, the \df{stochastic/transition kernel} from $T$ to $S$ is a function $\kappa: T \times \mathcal S \to [0,1]$ that satisfies: \begin{enumerate}
    \item for each $x \in T$, the function $\kappa (x,\blank)$ is a probability measure on $(S,\mathcal S)$;
    \item for each $A \in \mathcal S$, the function $x \mapsto \kappa (x,A)$ is $\mathcal T$-measurable.
\end{enumerate}
\end{defn}

Note that a transition kernel $\nu$ from $\Omega$ to $S$ is just a random measure. Now rephrasing \cref{def:reg-cond-dist}, the regular conditional distribution of $Y$ given $\mathcal G$ is a transition kernel $\nu$ from $\Omega$ to $S$ such that \begin{equation*}
    \text{the function }\omega \mapsto \nu(\omega,A) \text{ is a version of } P(Y \in A\giv \G),
\end{equation*} for each $A \in \mathcal S$.

If we consider two random variables $X\colon (\Omega,\F) \to (T,\mathcal T)$ and $Y\colon (\Omega,\F) \to (S,\mathcal S)$, respectively, then the regular conditional distribution of $Y$ given $X$ is $\kappa \circ X$, where $\kappa \colon S \times \mathcal T \to [0,1]$ is a transition kernel such that \begin{equation}\kappa(X(\omega),A) \text{ is a version of } P(Y \in A \giv X) (\omega). \label{eq:transition-ker-to-rcd} \end{equation} for each $A \in \mathcal S$.

It turns out quite surprising that the notion of a transition kernel fully generalizes \cref{def:reg-cond-dist}. Indeed it is clear that random measures are just transition kernels. To recover the definition of regular condition probability in \cref{def:reg-cond-dist}, we may set $X$ to be the identity map from $(\Omega,\G)$ to itself.

These are all formal definitions. When doing concrete computations, 

It turns out that regular condition distributions do not always exist. 

However, if we assume that $Y$ takes value in a standard Borel space $(S,\mathcal S)$, then the regular conditional probability of $Y$ given $X$ exists.

\begin{thm}
    % For an $(E,\mathcal E)$-valued random variable $X$ and another $(F,\F)$-valued random variable $Y$. Suppose $(F,\mathcal F)$ is a standard Borel space, then there exists a regular conditional distribution of 
    Let $\rho$ be a probability measure on the product space $(T \times S,\mathcal T \otimes \mathcal S)$, where $(S,\mathcal S)$ is a standard Borel space. Then $\rho = \mu \otimes \kappa$, where $\mu = \rho(\blank \times S)$ and $\kappa$ is a transition kernel from $T$ to $S$.
\end{thm}

    \begin{enumerate}
        \item For two discrete random variables $X$ and $Y$, we want \[\kappa(x,A) = \begin{cases}
            P(Y \in A \giv X = x) & \text{if } P(X = x) > 0 , \\
            \delta_{y_0}(A) & \text{if } P(X = x) = 0.
        \end{cases}  \]
        \item For two continuous random variables $X \in \R^m$ and $Y \in \R^n$, with joint density $f(x,y)$. We know the marginal density of $X$ is given by 
        \item Gaussian 
    \end{enumerate}

\cite[Theorem~8.5]{Kallenberg_2021}

\begin{namedthm}[Disintegration of random variables]
    Consider two random variables $X\colon (\Omega, \G) \to (T,\mathcal T)$ and $Y\colon (\Omega,\F) \to (S,\mathcal S)$, where $(S,\mathcal S)$ is a standard Borel space. Then the joint distribution $\rho$ of $(X,Y)$ is equal to the product measure $\mu \times \kappa$, where $\mu$ is the marginal distribution of $X$ and $\kappa$ is a transition kernel that satisfies \eqref{eq:transition-ker-to-rcd}.

    It follows that for any measurable $f \geq 0$ or $\E \abs{f(X,Y)} < \infty$, we have 
    \[
        \E [f(X,Y)\giv X] = \int f(X,y)\,\kappa(X,dy).
    \]
\end{namedthm}
\[
    \E [f(X,Y)\giv \G] = \int f(X,y)\,\nu(dy).
\]

\[
    \E f(X,Y) = \E \int f(X,y)\,\kappa(X,dy).
\]


\section{Stopping times}
A \df{discrete filtration} on a given a probability space $(\Omega,\F,P)$ is an expanding sequence of sub-$\sigma$-fields $\F_0 \subseteq \F_1\subseteq \dotsb$ of $\F$. Given a sequence of random variables $X_0, X_1, \dotsc$. we define its \df{natural filtration} by setting $\F_n = \sigma(X_0,X_1,\dotsc)$ for all $n\in \N_0$. 

\begin{xca}
Let $S$ and $T$ be two stopping times. Prove the following claims.
    \begin{enumerate}
        \item $S\bmin T$ and $S \bmax T$ are both stopping times.
        \item If $S \leq T$, then $\F_S \subseteq \F_T$.
        \item $\F_{S\bmin T} = \F_S \cap \F_T$.
    \end{enumerate}
\end{xca}

\begin{namedthm}[Wald's equations] \leavevmode
    \begin{enumerate}
        \item Let $X_1,X_2,\dotsc$ be i.i.d.\ $L^1$ random variables. If $T$ is a stopping time with $\E T<\infty$, then $\E (X_1 + \dotsc + X_T) = \E X_1 \E T$.
        \item Let $X_1,X_2,\dotsc$ be i.i.d.\ mean zero $L^2$ random variables.  If $T$ is a stopping time with $\E T<\infty$, then $\E (X_1 + \dotsc + X_T)^2 = \E(X_1)^2\E T$.
    \end{enumerate}
\end{namedthm}

\section{Martingales in discrete time}
Given a filtration $\{\F_n\}$, a \df{discrete martingale} is a sequence of $L^1$ random variables $X_n$, adapted to $\F_n$, such that \[
    \E(X_{n+1}\giv \F_n) = X_n.
\]

\begin{exa} Here are some of the most important examples of martingales coming up in applications. Let $\{X_j\}_j$ be a sequence of random variables, and $S_n = \sum_{j = 1}^n X_j$. Let $\F_n$ be the natural filtration with respect to $X_n$.
    \begin{enumerate}
        \item Linear martingales: if the sequence $\E X_j = 0$ for all $j$, then $S_n$ is a martingale.
        \item Quadratic martingales: if $\E X_j = 0$ and $\E X_j^2 = \sigma^2$ for all $j$, we have $S_n^2 - n\sigma^2$ as a martingale.
        \item Exponential martingales: say an unrelated sequence $Y_j$'s are nonnegative, i.i.d., with $\E Y_j = 1$, then $M_n = \prod_{j=1}^n Y_j$ is a martingale.

        It is clear that  $\frac{\exp(tX_j)}{M_{X_j}(t)}$ is a candidate for our $Y_j$.
    \end{enumerate}
\end{exa}

\begin{xca}
    Let $\{X_n\}$ be a martingale (resp.\ supermartingale, submartingale). For every $0 \leq n \leq m$, $\E(X_m \giv \F_n) = X_n$ (resp.\ $\leq$, $\geq$).
\end{xca}

\begin{prop}[(Martingale transformations under convex functions)]
    Let $\{X_n\}$ be adapted. For a convex $\phi\colon \R \to \R$ such that $\E \abs{\phi(X_n)} < \infty$, we have \begin{enumerate}
        \item if $\{X_n\}$ is a martingale, then $\{\phi(X_n)\}$ becomes a submartingale.
        \item if $\{X_n\}$ is a submartingale (resp.\ supermartingale), and $\phi$ is in addition increasing (resp.\ decreasing), then $\{\phi(X_n)\}$ remains a submartingale (resp.\ supermartingale)
    \end{enumerate}
\end{prop}

\begin{defn}
    A sequence of random variables $\{H_n\}$ is a predictable sequence if the sequence is bounded and each $H_{n+1}$ is $\F_n$ measurable.
\end{defn}

The \df{discrete stochastic integral} from time $0$ to $n\in \N_0$ is defined by \[
    (H \sbullet X)_{n} = H_1(X_1 - X_0) + H_2(X_2 - X_1) + \dotsc + H_n(X_n - X_{n-1}),
\] for $n \geq 1$, and $(H \sbullet X)_{0} = 0$.

It is useful to see that $(H\sbullet -X)_n = -(H \sbullet X)_n$.

\begin{prop} \leavevmode
\begin{enumerate} 
    \item If $\{X_n\}$ is a martingale, then $\{(H \sbullet X)_n\}$ is a martingale.
    \item If $\{X_n\}_n$ is a submartingale (resp.\ supermartingale), and $H_n \geq 0$ for all $n$, then $\{(H \sbullet X)_n\}$ is a submartingale (resp.\ supermartingale).
\end{enumerate}
\end{prop}

\begin{namedthm}[Optional stopping theorem, basic version] \label{thm:OST-1}
Let $\{X_n\}$ be a martingale (resp.\ supermartingale), and $T$ be a stopping time, both with respect to $\{\F_n\}$, then 
    \begin{enumerate}
        \item the stopped process $\{X_{n \bmin T}\}$ remains a martingale (resp.\ supermartingale);
        \item moreover, if $T \leq M$ a.s. for some $M < \infty$ (bounded stopping time), then $\E X_T = \E X_0$ (resp.\ $\leq \E X_0$).
    \end{enumerate}
\end{namedthm}

\begin{namedthm}[Doob's decomposition] \label{thm:doob-decomp}
    Any adapted integrable process $\{X_n\}$ can be uniquely decomposed by $X_n = M_n + A_n$, where $\{M_n\}$ is a martingale and $\{A_n\}$ is a predictable sequence starting from $A_0 = 0$. This is known as the \df{Doob decomposition}.

    Furthermore, an adapted integrable process $\{X_n\}$ is a submartingale (resp.\ supermartingale) if and only if it has a Doob decomposition with an increasing (resp.\ decreasing) predictable sequence.
\end{namedthm}
\begin{proof}
    The proof is quite elementary and may be left as an exercise. We want $X_n = M_n + A_n$, and conditioning both sides on $\F_{n-1}$ gives \begin{align*}
        \E(X_n \giv \F_{n-1}) = M_{n-1} + A_n = X_{n-1} - A_{n-1} + A_n.
    \end{align*}
    This gives for all $n\in \N$, \begin{equation}
        A_n - A_{n-1} = \E(X_n \giv \F_{n-1}) - X_{n-1}. \label{eq:doob-decomp}
    \end{equation} Set $A_0 = 0$, and by repeatedly applying the above identity we get \[
        A_n = \sum_{k=1}^n \E(X_k - X_{k-1} \giv \F_{k-1}) 
    \] that is $\F_{k-1}$ measurable. We have shown that the decomposition, if exists, must be unique.
    
    It remains to check that $M_n$ is indeed a martingale: \begin{align*}
        \E(M_n \giv \F_{n-1}) & = \E(X_n \giv \F_{n-1}) - A_n \\
            & = X_{n-1} - A_{n-1} = M_n,
    \end{align*} where we used \eqref{eq:doob-decomp}.

    The furthermore part follows immediately from \eqref{eq:doob-decomp}.
\end{proof}

\begin{namedthm}[Martingale convergence theorem] \label{thm:martingale-conv-thm}
    Say $\{X_n\}$ is a submartingale bounded in $L^1$, then the sequence $X_n$ converges a.s.\ to some $X_\infty \in L^1$.

    supermartingale/martingale 
\end{namedthm}

It is clear that the limit $X_\infty$ cannot be explicitly computed, and we have to employ some clever trick to prove the existence of the limit. 

\begin{lem}
    A sequence of real numbers $x = \{x_n\}$ converges if and only for any two rationals $a < b$, we have $U_\infty([a,b],x) < \infty$.
\end{lem}

\begin{namedthm}[Doob's upcrossing inequality]
    Let $X = \{X_n\}$ be a submartingale. Then for every $a < b$ and every $n \in \N$
    \[
        (b - a)\E U_n([a,b],X) \leq \E(X_n - a)^+ - \E(X_0 - a)^+.
    \]
\end{namedthm}

\begin{proof}[Proof of the \flcnameref{thm:martingale-conv-thm}]
    
\end{proof}

\nameref{thm:OST-1} may fail when the stopping time $T$ is unbounded.

The same example also show that the \flcnameref{thm:martingale-conv-thm} does not hold in the $L^1$ sense.

\section{Uniformly integrable martingales}

\begin{prop} \label{prop:cond-expec-unif-int}
    The collection $\{\E(X \giv \G) : \G \text{ is a sub-$\sigma$-field of }\F\}$ is uniformly integrable.
\end{prop}

\begin{thm}[(characterizations of uniformly integrable martingales)]
    For an $\F_n$-adpated martingale $X_n$, the following are equivalent. 
    \begin{enumerate}
        \item $\{X_n\}$ is uniformly integrable;
        \item $X_n$ converges a.s.\ and in $L^1$;
        \item $X_n$ converges in $L^1$;
        \item there exists an integrable $X$ such that $X_n = \E(X \giv \F_n)$.
    \end{enumerate}
\end{thm}
\begin{proof}
    (d)$\implies$(a) follows from \cref{prop:cond-expec-unif-int}. (b)$\implies$(c) is trivial. (a)$\implies$(b) is true because $\{X_n\}$ is bounded, and hence we may apply the \flcnameref{thm:martingale-conv-thm}.

    (c)$\implies$(d) is also not difficult. Let $X$ be the $L^1$ limit of $X_n$. Then for any $m > n$, we have $\E(X_m \giv \F_n) = X_n$. If we can show that $\E(X_m \giv \F_n) \to \E(X\giv \F_n)$ in $L^1$, then the proof is complete.

    \[
        \E\bigl\vert\E(X_m \giv \F_n) - \E(X\giv \F_n)\bigr\rvert \leq \E\bigl[ \E\bigl(\abs{X_m - X} \bigm\vert \F_n\bigr) \bigr]\leq\E\abs{X_m - X},
    \] which goes to $0$ as $m \to \infty$, as desired.
\end{proof}

\begin{namedthm}[Levy's zero--one law]
    Let $\{\F_n\}$ is a filtration with $\F_\infty = \sigma(\cup_n \F_n)$, which we write as $\F_n \uparrow \F_\infty$. Suppose $\E \abs{X}<\infty$, then \[
        \E(X \giv \F_n) \to \E(X\giv \F_\infty) \quad \text{a.s. and in }L^1.
    \]

    In particular, for $A \in \F_\infty$, we have \[
        \E(\ind_A \giv \F_n) \to \ind_A \quad \text{a.s. and in }L^1.
    \]
\end{namedthm}

\begin{namedthm}[DCT for conditional expectations]
    Suppose $X_n \to X$ a.s.\ and for all $n \in \N_0$, $\abs{X_n} \leq Y$ for some $Y \in L^1$. Given that the $\sigma$-fields $\F_n \uparrow \F_\infty$, then \[
        \E(X_n \giv \F_n) \to \E(X \giv \F_\infty).
    \]

    
\end{namedthm}

\section{Backward martingales and their applications}
\begin{defn}
    A \df{backward filtration} is a $\Z^{\leq 0}$-indexed filtration, i.e., a sequence of sub-$\sigma$-fields of $\F$ \[
        \dotsb\subseteq \F_2 \subseteq \F_1 \subseteq \F_0.
    \] Let $\F_\infty = \bigcap_{n=-\infty}^0 \F_n$, which we know is again a sub-$\sigma$-field of $\F$.
\end{defn}

\begin{namedthm}[Backward martingale convergence theorem]
    The sequence $X_n \to X_{-\infty}$ a.s.\ and in $L^1$.
\end{namedthm}

\begin{exa}[(another proof of the \nameref{thm:SLLN})]
    
\end{exa}

\begin{exa}[(another proof of the \nameref{thm:HS-01-law})]
    
\end{exa}

\section{\texorpdfstring{$L^p$}{Lp} convergence of martingales}

\begin{namedthm}[Doob's maximal inequality]
    Let $\{X_n\}$ be a submartingale, then for every $a > 0$, we have \[
        a P\bigl(\max_{0\leq k \leq n} X_k \geq a\bigr) \leq \E\Bigl(X_n \ind\bigl\{\max_{0\leq k \leq n}X_n \geq a\bigr\}\Bigr) \leq \E X_n^+.
    \]
    If $\{Y_n\}$ is a supermartingale, then  for every $a > 0$, we have \[
        a P\bigl(\max_{0\leq k \leq n} Y_k \geq a\bigr) \leq \E Y_0 + \E Y_n^-.
    \]
    Combining the two cases above, we get for a submartingale or supermartingale $\{X_k\}$, it holds that \[
        a P\bigl(\max_{0\leq k \leq n} \abs{X_k} \geq a\bigr) \leq \E \abs{X_0} + 2 \E \abs{X_n}.
    \]
\end{namedthm}

The technique of introducing an appropriate stopping time

Integrating the two sides of the first inequality, we can obtain an $L^p$ moment bound on $\E(\max X_k)$ for $1 < p < \infty$.

\begin{namedthm}[Doob's $L^p$ inequality] \label{thm:Doob-Lp-discrete}
    Let $1 < p < \infty$ and $\{X_n\}$ be a nonnegative submartingale. For each $n\in \N_0$, we have \[
        a^p P\bigl(\max_{0\leq k \leq n} X_k \geq a\bigr) \leq \E \bigl(\max_{0\leq k \leq n} X_k\bigr)^p \leq \biggl(\frac{p}{p-1}\biggr)^p \E(X_n)^p.
    \]

    Therefore if $\{Z_n\}$ is a martingale, then $\{\abs{Z_n}\}$ is a nonnegative submartingale. Therefore we have \[
        a^p P\bigl(\max_{0\leq k \leq n} \abs{Z_k} \geq a\bigr) \leq \E \bigl(\max_{0\leq k \leq n} \abs{Z_k}\bigr)^p \leq \biggl(\frac{p}{p-1}\biggr)^p \E\abs{Z_n}^p.
    \]
\end{namedthm}


We say $\{X_n\}$ is a \df{square integrable martingale} if $\{X_n\}$ is a martingale, and each $X_n \in L^2(P)$.

When $\{X_n\} \subseteq L^2$ is a martingale, then we have by \nameref{thm:Doob-Lp-discrete} ($p=2$) that \begin{equation}
    \E \max_{0\leq k \leq n} X_k^2  \leq 4 \E X_n^2. \label{eq:apply-L2-maximal}
\end{equation}
We now show that a uniform control on $\{X_n\}_{n\in \N}$ can be obtained.
\nameref{thm:doob-decomp} tells us that we can decompose the submartingale $X_n^2$ into $M_n + A_n$, where $\{M_n\}$ is a martingale, and $\{A_n\}$ is an increasing predictable sequence given by \begin{align*}
    A_n & = \sum_{k=1}^n \E(X_k^2 - X_{k-1}^2 \giv \F_{k-1}) \\
    & = \sum_{k=1}^n \E(X_k^2 - 2 X_k X_{k-1} + X_{k-1}^2 \giv \F_{k-1})\\
    & = \sum_{k=1}^n \E[(X_k- X_{k-1})^2 \giv \F_{k-1}],
\end{align*}
where we have used $\E(X_k \giv \F_{k-1}) = X_{k-1}$ in the second equality. This increasing sequence has a special name, called the \df{quadratic variation} of the square integrable martingale $\{X_n\}$, which we denote by $\{\qv{X}_n\}$.

Note $\E X_n^2 = \E M_n + \E \qv{X}_n = \E X_0^2 + \E \qv{X}_n$, and we may plug this into \eqref{eq:apply-L2-maximal}. By the monotone convergence theorem, we can therefore conclude 

\begin{prop}
    For martingale $\{X_n\} \subseteq L^2$, we have \[
    \E \sup_n  X_n^2 \leq  4 \E \qv{X}_\infty + 4 \E X_0^2,
\] where $\qv{X}_\infty = \lim_n \qv{X}_n$, which is possibly infinite.
\end{prop}


\begin{namedthm}[$L^p$ convergence theorem for martingales]
    Let $1<p<\infty$, and $\{X_n\}$ be a uniformly $L^p$-bounded martingale. Then $X_n$ converges a.s.\ and in $L^p$ to some $X_\infty$ satisfying \[
        \E \abs{X_\infty}^p = \sup_n \E\abs{X_n}^p.
    \] Meanwhile \[
        \E \bigl(\sup_n\abs{X_n}\bigr)^p \leq \biggl(\frac{p}{p-1}\biggr)^p \E\abs{X_\infty}^p.
    \]

    \begin{thm}[(convergence of $L^2$ summable random series)]
    Let $\{X_n\}$ be a sequence of independent mean zero $L^2$ random variables, then the following are equivalent: \begin{enumerate}
        \item $\sum_{n=1}^\infty \E X_n^2 < \infty$;
        \item $\sum_{n=1}^\infty X_n^2$ converges a.s.\ and in $L^2$;
        \item $\sum_{n=1}^\infty X_n^2$ converges in $L^2$.
    \end{enumerate}
\end{thm}
\end{namedthm}

% \section{Square integrable martingales}

\section{Martingales of bounded increments}
\begin{thm}[(convergence behavior)]
    For a martingale $\{X_n\}$ with $\sup_n \abs{X_{n+1} - X_n} < \infty$, we have almost surely either $\lim_n X_n$ exists and is finite, or $\limsup_n X_n= +\infty$ and $\liminf_n X_n = -\infty$. 
\end{thm}

\begin{fact}
    For a martingale $\{X_n\}$, we have for any Borel measurable function $f$ that \[
        \E\bigl[(X_{n+1}-X_n)f(X_0,X_1,\dotsc,X_{n})\bigr] = 0
    \] by the \flcnameref{thm:tower}. In particular, we have \[
        \E\bigl[(X_{n+1}-X_n)(X_{m+1} - X_m)\bigr] = 0
    \] for any $n > m$, i.e., martingale differences are uncorrelated.
\end{fact}

\begin{namedthm}[Azuma--Hoeffding inequality] \label{thm:Azuma-Hoeffding}
    Let $\{X_n\}$ be a supermartingale, and $\{A_n\}$ and $\{B_n\}$ are predictable with respect to the filtration $\{\F_n\}$, such that \[A_n \leq X_n - X_{n-1} \leq B_n.\] If for all $A_n$ and $B_n$ we have some positive constant $c_n$ such that $B_n - A_n \leq c_n$, then we have \begin{equation} \label{eq:Azuma-Hoeffding-supmg}
        P(X_n - X_0 \geq t) \leq \exp\biggl(-\frac{2 t^2}{\sum_{k=1}^n c_k^2}\biggr).
    \end{equation}

    If the $\{X_n\}$ above is a submartingale instead, then we get \[
        P(X_0 - X_n \geq t) \leq \exp\biggl(-\frac{2 t^2}{\sum_{k=1}^n c_k^2}\biggr).
    \] Hence by a simple union bound, we get for a martingale $\{X_n\}$ with the described conditions, it holds that \[
        P(\abs{X_n - X_0} \geq t) \leq 2\exp\biggl(-\frac{2 t^2}{\sum_{k=1}^n c_k^2}\biggr).
    \]
\end{namedthm}
\begin{proof}
    We first show \eqref{eq:Azuma-Hoeffding-supmg} when $\{X_n\}$ is only a martingale, and at the end we extend the inequality to the supermartingale case by invoking \nameref{thm:doob-decomp}.
    
    The Chernoff method is expected, just by observation of the inequality. For any $\lambda \in\R$, we have 
\begin{equation}
P(X_{n}-X_{0}\geq t)\leq\frac{\E\exp\bigl(\lambda (X_{n}-X_{0})\bigr)}{e^{\lambda t}}.\label{eq:chernoff}
\end{equation}
Now focus on $\E\exp\bigl(\lambda (X_{n}-X_{0})\bigr)$, which is equal to
\begin{align}
\E\exp\biggl(\lambda \sum_{k=1}^{n}X_{k}-X_{k-1}\biggr) & =\E\biggl[\exp \lambda (X_{n}-X_{n-1})\cdot\exp\Bigl(\lambda\sum_{k=1}^{n-1}X_{k}-X_{k-1}\Bigr)\biggr]\nonumber \\
 & =\E\biggl[\E\Bigl(\exp \lambda(X_{n}-X_{n-1})\cdot\exp\Bigl(\lambda \sum_{k=1}^{n-1}X_{k}-X_{k-1}\Bigr)\Bigm\vert\F_{n-1}\Bigr)\biggr]\nonumber \\
 & =\E\biggl[\exp\Bigl(\lambda \sum_{k=1}^{n-1}X_{k}-X_{k-1}\Bigr)\E\Bigl(\exp\bigl(\lambda(X_{n}-X_{n-1})\bigl)\Bigm\vert\F_{n-1}\Bigr)\biggr].\label{eq:one-step}
\end{align}
Since $\{X_{n}\}$ is an $\{\F_{n}\}$-adapted martingale, for the difference sequence $Y_{n}=X_{n}-X_{n-1}$, we should have 
\[
\E(Y_{n}\giv\F_{n-1}) = 0.
\]
Also by assumption for constant $c_{n}>0$ and random variable
$A_{n}$ that is $\F_{n-1}$-measurable, we have 
\[
A_{n}\leq Y_{n}\leq A_{n}+c_{n},
\]
therefore by the conditional version of \nameref{lem:Hoeffding}, line~\eqref{eq:one-step} is 
\[
\leq \E\biggl[\exp\Bigl(\lambda \sum_{k=1}^{n-1}X_{k}-X_{k-1}\Bigr)\biggr]
\exp\biggl(\frac{\lambda^{2}c_{n}^{2}}{8}\biggr).
\]

We may repeat the procedure above by first conditioning and then applying
Hoeffding's lemma, and obtain in the end that 
\begin{align*}
\E\exp\biggl(\lambda \sum_{k=1}^{n}X_{k}-X_{k-1}\biggr) & \leq \prod_{k=1}^{n}\exp\biggl(\frac{\lambda^{2}c_{k}^{2}}{8}\biggr)\\
 & =\exp\biggl(\frac{\lambda^{2}\sum_{k=1}^{n}c_{k}^{2}}{8}\biggr).
\end{align*}
Going back to \eqref{eq:chernoff}, we have 
\[
P(X_{n}-X_{0}\geq t) \leq \exp\biggl(\frac{\lambda^{2}\sum_{k=1}^{n}c_{k}^{2}}{8}-\lambda t\biggr).
\]
The quadratic expression $\frac{\sum_{k=1}^{n}c_{k}^{2}}{8}\lambda^{2}-t\lambda$ in $\lambda$ has minimum value 
\[
-\frac{t^{2}}{4\cdot\frac{\sum_{k=1}^{n}c_{k}^{2}}{8}}=-\frac{2t^{2}}{\sum_{k=1}^{n}c_{k}^{2}}.
\] when $\lambda =-\frac{-t}{2\cdot\frac{\sum_{k=1}^{n}c_{k}^{2}}{8}}$.
Therefore 
\[
P(X_{n}-X_{0}\geq t)\leq\exp\biggl(-\frac{2t^{2}}{\sum_{k=1}^{n}c_{k}^{2}}\biggr),
\]
finishing the proof for the martingale case.

For a supermartingale $\widehat{X}_n$, we know by \nameref{thm:doob-decomp} that there is a (unique) decomposition $\widehat{X}_n = X_n + D_n$, where $X_n$ is a martingale and $D_n$ is a decreasing sequence. This implies that \begin{align*}
    P(\widehat{X}_n - \widehat{X}_0 \geq t) & = P(X_n - X_0 + D_n - D_0 \geq t)\\
    & \leq P(X_n - X_0 \geq t),
\end{align*} as $D_n - D_0 \leq 0$. The proof is now complete.
\end{proof}

It is clear that \nameref{thm:Hoeffding-ineq} is simply a special case of the above martingale inequality. Another applicable consequence of \nameref{thm:Azuma-Hoeffding} is the following result about concentration of functions taking vector inputs of independent components.

\begin{namedthm}[McDiarmid’s bounded difference inequality]
\label{thm:bdd-diff} Let a measurable function $g\colon\prod_{k=1}^n S_k\to\R$ satisfy
the bounded difference property with constants $c_{1},\dotsc,c_{n}$. This means for each $k\in[n]$, we have 
\[
\sup_{\substack{x_{1},\dotsc, x_{n}\\x_{k}'\in S_k}}\abs{g(x_{1},\dotsc x_{n})-g(x_{1},\dotsc x_{k-1},x_{k}',x_{k+1},\dotsc,x_{n})}\leq c_{k}.
\]
Let $\{X_{k}\}_{k=1}^n$ be independent $S_k$-valued random variables, then 
\[
P\bigl(g(X_{1},\dotsc,X_{n})-\E g(X_{1},\dotsc,X_{n})\geq t\bigr)\leq\exp\biggl(\frac{-2t^{2}}{\sum_{j}c_{j}^{2}}\biggr).
\]

(Note that by the triangle inequality, $g(\mathbf x) - g(\mathbf x') \leq \sum_{k=1}^n c_k$ for any $\mathbf x ,\mathbf x' \in \prod_{k}S_k$, and hence the function $g$ is bounded.)
\end{namedthm}
\begin{proof}
    Define the martingale $Y_k = \E [g(X_1,\dotsc,X_n) \giv \F_k]$ for $0\leq k \leq n$, where $\F_k$ is the natural filtration with respect to $\{X_k\}$. It suffices to check the conditions of \nameref{thm:Azuma-Hoeffding}. Note $Y_k - Y_{k-1}$  can be expanded into \begin{equation} \label{eq:expand-mg-bdd-diff}
        \E [g(X_1,\dotsc,X_{k},\xi_{k+1},\dotsc,\xi_{n}) - g(X_1,\dotsc,X_{k-1},\xi_{k},\dotsc,\xi_{n}) \giv \F_{k}], % \\ & \qquad - \E [g(X_1,\dotsc,X_{k},\xi_{k+1},\dotsc,\xi_{n} \giv \F_{k-1}]
    \end{equation} where $\xi_k,\dotsc,\xi_n$ are copies of $X_{k},\dotsc,X_n$ that are independent of everything else.

    Now we define our natural candidate for $A_k$ by \[
        \inf_{z \in S_k} \E [g(X_1,\dotsc,X_{k-1},z,\xi_{k+1}\dotsc,\xi_{n}) - g(X_1,\dotsc,X_{k-1},\xi_{k},\xi_{k+1}\dotsc,\xi_{n}) \giv \F_{k}],
    \] and let $B_k$ be the corresponding supremum; both are $\F_{k-1}$-measurable. The bounded difference condition gives that \begin{align*}
        & \phantom{{}={}}B_k - A_k\\ & = \sup_{z,w\in S_k} \E[g(X_1,\dotsc,X_{k-1},z,\xi_{k+1}\dotsc,\xi_{n}) - g(X_1,\dotsc,X_{k-1},w,\xi_{k+1}\dotsc,\xi_{n}) \giv \F_k]
        \\ & \leq c_k,
    \end{align*} and hence we have verified all the conditions for invoking Azuma--Hoeffding.
\end{proof}

\begin{namedthm}[Efron--Stein inequality]
    Define $\Var_j(x_1,\dotsc,x_n) = \Var(x_1,\dotsc,X_j,\dotsc,x_n)$. For independent $X_1,\dotsc,X_n$, we have \[
        \Var f(X_1,\dotsc,X_n) \leq \E \biggl(\sum_{j=1}^n \Var_j f(X_1,\dotsc,X_n)\biggr).
    \]
\end{namedthm}

\section{Gamblers' ruin and random walks}

For a random process $\{X_t\}$ that starts $X_0 = x$, we often use $\bP_x$ instead of $P$ as the notation for the underlying probability measure. \emph{There is absolutely difference between the two in the context of this section.} The subscript $x$ here is merely used to emphasize where the random process starts. However, it deserves attention that $\bP_x$ is a distinct probability measure living on a different space that is induced from the usual $P$ on $(\Omega, \F)$. We will discuss this at a detailed level in the upcoming chapter, when discussing the canonical probability space for a Markov chain.

\begin{thm}
    Let $S_n$ be the symmetric random walk on $\Z$ that starts at $0$, and define $T = \min\{n  : S_n \notin (-a,b)\}$, where $-a < 0 \leq b$ are integers. We have $T< \infty$ a.s., and \[
        \bP_0(S_T = - a) = \frac{b}{a+b},\ \bP_0(S_T = b) = \frac{a}{a+b}, \text{ and } \bE_0 T = ab.
    \]
\end{thm}

Take $-a = -1$ and $b = N-1 \geq 0$. For any $y\in \N$, define the hitting time $T_y = \min\{n : S_n = y\}$, then \[
    \bP_0(T_{-1} < T_{N-1}) = \frac{N - 1}{N} \text{ and }\bP_0(T_{-1} > T_{N-1}) = \frac{1}{N}.
\] Therefore \[
    \bP_0(T_{-1} < \infty) = \bP_0\biggl(\bigcup_{N=1}^\infty \{T_{-1} < T_{N-1}\}\biggr) = 1.
\] However, $\bE_0 T_{-1} = \infty$.

by monotone convergence

\begin{thm}
    Let $S_n = \sum_{j=1}^\infty \xi_j$ be the asymmetric random walk that starts from $0$, where each $\xi_j$ is i.i.d., with $P(\xi_j =1) = p > 1/2$ and $P(\xi_j = -1) = q = 1 - p < 1/2$.

    \begin{enumerate}
        \item First, one can verify that $\bigl\{(q/p)^{S_n}\bigr\}_n$ is a martingale.
        \item Therefore let $f(y) = (q/p)^y$. Again define the hitting time $T_z = \min\{n : S_n = z\}$. For $- a < 0 < b$, we have\[
        \bP_0 (T_{-a} < T_b) = \frac{\phi(b) - \phi(0)}{\phi(b) - \phi(-a)} \text{ and } \bP_0 (T_{-a} > T_b) = \frac{\phi(0) - \phi(-a)}{\phi(b) - \phi(-a)}.
        \]
        \item Now we look at the two hitting times individually: \[\bP_0(\min_n S_n \leq -a) = \bP_0(T_{-a} < \infty) = \biggl(\frac{1 - p}{p}\biggr)^a.\]
        \[
        \bP_0(\max_n S_n \geq b ) = \bP_0(T_b < \infty) = 1 \quad \bE_0 T_b = \frac{b}{2p-1}.
    \]
    \end{enumerate}
    This stopping time conversion is very standard
\end{thm}