\chapter{Product spaces}
\section{Product \texorpdfstring{$\sigma$-algebras}{sigma-algebra}}

We start with a comparison between product topologies and product $\sigma$-algebras. See \cite[Sections 4.1 and 4.2]{folland1999} for a review of bases, subbases and product topologies.

For topological spaces $(X_\alpha,\mathcal{T}_\alpha)$ ($\alpha \in I$), recall that the \df{product topology} $\mathcal{T}$ on $X = \prod_{\alpha \in I} X_\alpha$ is the topology generated by all coordinate projections $\pi_\alpha\colon X \to X_\alpha$ (i.e., the smallest topology on $X$ that makes all these maps continuous). Explicitly $\mathcal T$ is generated by the collection of subbasic sets \begin{equation} \label{eq:1d-cylinder-top}
    \{\pi_\alpha^{-1}(U_\alpha): U_\alpha\in \mathcal{T}_\alpha, \alpha \in I\}.
\end{equation}
    
For measurable spaces $(X_\alpha, \A_\alpha)$ ($\alpha \in I$), the \df[product sigma-algebra@product $\sigma$-algebra]{product $\sigma$-algebra} $\A = \bigotimes_{\alpha\in I} \A_\alpha$ on $X = \prod_{\alpha \in I} X_\alpha$ is the $\sigma$-algebra generated by all coordinate projections $\pi_\alpha$. Explicitly $\A$ is generated by the collection of sets \begin{equation} \label{eq:1d-cylinder-sa}
     \{\pi_\alpha^{-1}(E_\alpha): E_\alpha\in \A_\alpha, \alpha \in I\}.
\end{equation}

Define the general \df[cylinder set]{cylinder sets}\footnote{This definition similarly holds for other set-collection pairs.} on the product of topological spaces $(X_\alpha,\mathcal{T}_\alpha)$ and measurable spaces $(X_\alpha,\A_\alpha)$ to be the sets of form \[
    \bigcap_{j=1}^n \pi_{\alpha_{j}}^{-1}(U_{\alpha_j}) \quad \text{and} \quad \bigcap_{j=1}^n \pi_{\alpha_{j}}^{-1}(E_{\alpha_j}),
\] for any $n \in \N$, respectively. To put them into simple words, they are finite intersections of preimages of the projections. The collection of sets in \eqref{eq:1d-cylinder-top} and \eqref{eq:1d-cylinder-sa} are $1$-dimensional cylinders.

The general cylinder sets on the product of topological spaces, as finite\footnote{As another reminder, if the intersection is allowed to be arbitrary, then we get a larger topology called the \df{box topology}. The box topology is generated by full-dimension products of open sets. When the product is finite, the box topology and the product topology coincide.} intersections of subbasic sets in \eqref{eq:1d-cylinder-top}, form a basis for the product topology $\mathcal{T}$. However, it is a well-known fact that $\sigma$-algebras, unlike topologies, cannot be written out explicitly from the elementary sets they are generated from. % Therefore in studying product $\sigma$-algebras, we should work primarily with $1$-dimensional cylinder sets of the form \[
%     E_\beta \times \prod_{\alpha \neq \beta} X_\alpha
% \] over all $\beta \in I$ and $E_\beta \in \A_\beta$.

Looking back at \eqref{eq:1d-cylinder-sa}, you may expect a smaller collection of cylinder sets generates the product $\sigma$-algebra. Yet the proof is a little weird, like most arguments involving algebras of sets.

\begin{prop} \label{prop:prod-s-algebra-generate}
    Suppose each $\A_\alpha$ is generated by $\mathcal{E}_\alpha$. Then $\bigotimes_{\alpha} \A_\alpha$ is generated by the collection \[
        \mathcal{K} = \{\pi_\alpha^{-1} (E_\alpha) : E_\alpha \in \mathcal{E}_\alpha, \alpha \in I\}.
    \]
\end{prop}
\begin{proof}
    Let the collection in \eqref{eq:1d-cylinder-sa} be $\mathcal{J}$. Clearly $\mathcal{K} \subseteq \mathcal{J}$. To see the other inclusion, consider the induced $\sigma$-algebra on $X_\alpha$ \[
        \{E \subseteq X_\alpha : \pi_\alpha^{-1}(E) \in \sigma(\mathcal{K})\},
    \] which contains $\mathcal{E}_\alpha$ and hence $\A_\alpha$. This means $\pi_\alpha^{-1}(E) \in \sigma(\mathcal{K})$ for all $\alpha \in I$ and $E \in \A_\alpha$. Hence $\mathcal{J} \subseteq \sigma(\mathcal{K})$. The proof is now complete.
\end{proof}

We have introduced very general definitions above, but in practice we mostly deal with cases where the index set $I$ is countable. The reader should verify on their own that when $I$ is countable, $\A = \bigotimes_{k=1}^\infty \A_k$ is generated by \[
    \biggl\{\prod_{k=1}^\infty E_k :E_k \in \A_k \biggr\}.
\] Also, for measurable spaces $(X_1,\A_1),(X_2,\A_2),\dotsc$, the product $\sigma$-algebra $\A$ is clearly generated from cylinder sets of the form \[
    I_{n,B} = B \times \prod_{k= n+ 1}^\infty X_n\text{, where }B \in \bigotimes_{k=1}^n \A_k.
\] This turns out to be clean to work with.


3.5.1 3.5.2 Bogachev

Since the Borel $\sigma$-algebra is the $\sigma$-algebra generated by open set, while the topological space consists of all the open sets. With our above detailed comparisons between product $\sigma$-algebras and product topological spaces, the Borel $\sigma$-algebra from the product topology and the product Borel $\sigma$-algebra from individual spaces should be the same, under some conditions.

\begin{thm}
    For any separable metric spaces $X_1,X_2,\dotsc$ (finite or countably infinite), we have \begin{equation} \label{eq:borel-prod-agreement}
        \B(X) = \B(X_1) \otimes \B(X_2) \otimes \dotsb,
    \end{equation} where $X = X_1 \times X_2 \times \dotsb$ with product topology $\mathcal{T}$ given by the supremum metric.
\end{thm}
\begin{proof}
    We follow the proof in \cite{Kallenberg_2002}\footnote{This is the second edition of the book. The new proof in the third edition is very misleading, and I suspect there are many errors in the new edition.}.
    
    Let $\mathcal{J}$ be the class of $1$-dimensional cylinder sets \[
        X_1\times \dotsb \times X_{k-1} \times U_k \times X_{k+1} \times \dotsb
    \] over all $k \in \N$ and $U_k \in \mathcal{T}_k$.
    
    Since $\mathcal{J}$ consists entirely of open sets, and $\text{RHS} = \sigma(\mathcal{J})$ by \cref{prop:prod-s-algebra-generate}, we have $\text{LHS} \supseteq \text{RHS}$. \emph{Note that this inclusion does not use any topological assumptions on the $X_n$'s.}
    
    If we can now show that $\mathcal{T} \subseteq \sigma(\mathcal{J})$, the proof will be complete. Now $(X,\mathcal{T})$, as a product of separable metric spaces, is still a separable metric space. Here we use a result from the \cite{Bogachev_2020}, included as \cref{prop:Lind-property-sep-metric-space} in the appendix:
    
    \begin{center}
    \noindent\begin{minipage}[t]{0.9\columnwidth}
        Every collection of open sets in a separable metric space contains an at most countable subcollection with the same union.
    \end{minipage}
    \end{center}
    
    Therefore every open set in $X$ is a countable union of basic open sets. Since a topological basis is given by finite intersections of the cylinder sets in $\mathcal{J}$, we then have $\mathcal{T} \subseteq \sigma(\mathcal{J})$.
    % For the countable dense subset $D_k$ of $X_k$, we 
    % Let $\mathcal{T}$ be the product topology on $X_1 \times X_2 \times \dotsb$.Then \[
    %     \text{LHS} = \B(\mathcal{T}).
    % \] Let $\mathcal{C}$ be the collection of $1$-dimensional cylinder sets of the form \[
    %     X_1 \times \dotsb\times X_{k-1}  \times E_k \times X_{k+1} \times \dotsb,
    % \] where $E_k \in \mathcal{T}_k$. Then \[
    %     \text{RHS} = \B(\mathcal{K}).
    % \]
\end{proof}

The direct corollary is that $\B(\R^d) = \bigotimes^d \B(\R^1)$. This theorem overall shows the fundamental importance of Borel $\sigma$-algebra in measure theory and its applications: it connects measurability to the underlying topological spaces.

As an exercise, use \cref{prop:measurability-generate} to show the following: 
\begin{xca}[{\cite[Proposition~2.4]{folland1999}}]
    Given measurable spaces $(X,\M)$ and $(Y_\alpha,\mathcal{N}_\alpha)$ over all $\alpha \in I$. Let $Y = \prod Y_\alpha$ and $\mathcal{N} = \bigotimes \mathcal{N}_\alpha$. Then $f\colon X \to Y$ is $\M/\mathcal{N}$-measurable if and only if each $f_\alpha = \pi_\alpha \circ f$ is $\M/\mathcal{N}_\alpha$-measurable.
\end{xca}


We reserve the discussion of two extremely important existence results about probability measures on product spaces to \cref{sec:product-prob-meas}. The first of the two results () tells us that there is a \emph{natural} extension of product probability measures over all finite cylinder sets to a product probability measure over the entire product $\sigma$-algebra.
The second result () says that if a sequence of probability measures are specified in a \emph{consistent way}, then there is a natural extension of them to a product measure on the entire product $\sigma$-algebra. 

Note that it makes sense to only discuss the countable product of \emph{probability} measures, so that both the coordinate measures, the finite-dimensional product measures. and the countable product measures are all \emph{normalized}. Because of this, and the significance of the existence theorems for product measures in probability, we delay our discussion of these two results despite their purely measure-theoretic statements and proofs.

Many books in probability only include and applies it as a special case

\section{Integration on product spaces} \label{sec:prod-integrate}
\begin{namedthm}[Fubini--Tonelli theorem]
    
\end{namedthm}

folland exercise 12

\section{Change of variables} \label{sec:cov}
\section{Gamma functions and polar coordinates} 
Cauchy formula for repeated integration

\label{sec:polar}
Let $z \in \C$ with $\Re z > 0$, and we define $f_z\colon (0,\infty)\to \C$ by \[f_z(t) = t^{z-1} e^{-t} = \exp\bigl((z-1)\log t\bigr)\cdot e^{-t}.\] Since 

$\sigma(S^{n-1})=\frac{2\pi^{n/2}}{\Gamma(n/2)}$ and $m(B^{n})=\frac{1}{n}\sigma(S^{n-1})=\frac{\pi^{n/2}}{\Gamma\bigl(\frac{n}{2}+1\bigr)}$.
For any $\epsilon>0$, we have $S^{n-1}\subseteq B^{n}(0;1+\epsilon)-B^{n}(0;1)$
\begin{align*}
m(S^{n-1}) & \leq m\bigl(B^{n}(0;1+\epsilon)\bigr)-m\bigl(B^{n}(0;1)\bigr)\\
 & \leq(1+\epsilon)^{n}m(B^{n})-m(B^{n}).
\end{align*}
Take $\epsilon\to0^{+}$, it is easy to see that $m(S^{n-1})=0$.
surface area