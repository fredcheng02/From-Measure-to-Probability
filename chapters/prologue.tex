\phantomsection
\chapter*{\Large Prologue}
\addcontentsline{toc}{part}{Prologue}
\chaptermark{Prologue}

This is the most ambitious writing project undertaken by the author so far as a math student, and he hopes he can finish it in two years. The author, as a probability student, did not excel in his real analysis courses (MATH 202AB at UC Berkeley) during his senior year. To compensate, the author aims to write an extensive and detailed note that surveys through all the major measure theory results of interest to a rigorous-minded mathematical probabilist.

Part I of this note will be devoted to measure theory in a general setting, while Part II will discuss results in probability spaces built on top of Part I. The author hopes that his commentary and the overall structure of the survey can help the readers (and himself) truly understand both abstract measure theory and probability theory from a measure-theoretic point of view.

This entire survey will be based on multiple sources, listed in the bibliography page. As the old saying goes, ``if you copy from one book that is plagiarism, but if you copy from ten books that is scholarship.''
\vspace{1\baselineskip}

\noindent Shanghai, August 2024 \hfill F.C.

\vspace{3\baselineskip}
The prerequisite for this survey notes is a strong background in undergraduate real analysis and familiarity with elementary probability theory. Some key results about normed spaces, Hilbert spaces, and topology will be assumed, and these can usually be found on any first-year graduate analysis texts. Some rudimentary familiarity with weak topology on Banach spaces will contribute to the understanding of weak and vague convergence of measures. We have also included appendices at the end of the survey, which discuss some of these facts at a high level.

\emph{Remarks on Notation.} In Part I we use $X$ to denote a nonempty set, but in Part II we use $X$ instead to denote a random variable. As a replacement a nonempty set is denoted by $S$. Oftentimes $S$ is a metric space with metric $\rho$.
 
If you see any errors or typos, please inform the author via \begin{center}
    \href{mailto:fecheng@uw.edu}{\texttt{fecheng@math.washington.edu}}.
\end{center}