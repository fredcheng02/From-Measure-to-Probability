\chapter{A cornucopia of ergodic theory}
Given a probability space $(\Omega,\F,\mu)$, a \df{measure-preserving transformation} (MPT) $T$ is a measurable function from $(\Omega,\F)$ to itself such that \[
    \mu(T^{-1}A) = \mu(A)\text{ for all }A\in \F.
\] The resulting quartet $(\Omega,\F,\mu,T)$ is called a \df{measure-preserving dynamical system} (MPDS). If $T$ is invertible, and $T^{-1}$ is measurable, then it is equivalent to say $T$ is measure-preserving if \[
    \mu(TA) = \mu(A) \text{ for all }A\in \F.
\]

An MPT $T$ is said to be $\mu$-\df{ergodic} (or the measure $\mu$ is said to be $T$-\emph{ergodic}) if for all $A\in \F$, we have \[
    \mu(A \symdiff T^{-1} A) = 0 \implies \mu(A) = 0 \text{ or } 1.
\] A set $A\in \F$ satisfying $\mu(A \symdiff T^{-1} A) = 0$ is called \df[almost invariant@(almost) invariant]{(almost) invariant}. If instead we have $T^{-1}A = A$, then $T$ is \df{strictly invariant}. The ergodicity of $T$ can be equivalently defined by \[
    T^{-1} A = A \implies \mu(A) = 0 \text{ or } 1,
\] that is, we only need to check strictly invariant sets must be of measure $0$ or $1$.

One direction is obvious. For the other direction, one can check that for any set $A\in \F$, the set $B = \limsup_n T^{-n} A$ is always going to be strictly invariant.

\begin{defn}
    An MPDS $(\Omega,\F,\mu,T)$ is said to be \df[mixing!strong]{strong mixing} if for all $A,B\in \F$, \begin{equation}
        \lim_n \mu(A\cap T^{-n} B) = \mu(A) \mu(B); \label{eq:strong-mix}
    \end{equation}
    it is said to be \df[mixing!weak]{weak mixing} if for all $A,B\in \F$,\begin{equation} \label{eq:weak-mix}
        \lim_n \frac{1}{n} \sum_{k=0}^{n-1} \bigl\lvert \mu(A \cap T^{-k} B) - \mu(A) \mu(B) \bigr\rvert = 0,
    \end{equation} i.e., $\bigl\lvert \mu(A \cap T^{-k} B) - \mu(A) \mu(B) \bigr\rvert$ converges to $0$ in the Cesàro sense.
\end{defn}

Hence strong mixing implies weak mixing. In fact weak mixing further implies the system is ergodic. Let $A = B\in \F$ be strictly invariant, then we may replace $T^{-k} B$ by $B$ in \eqref{eq:weak-mix} and get $\mu(B) = \mu(B)^2.$

Notice that the above argument remains true if we remove the $\abs{\blank}$ in the definition \eqref{eq:weak-mix} of weak mixing. It turns out that \[
    \lim_n \frac{1}{n} \sum_{k=0}^{n-1} \mu(A \cap T^{-k} B) = \mu(A)\mu(B)\quad \text{for all }A,B\in \F
\] is in fact equivalent to the saying that the system is ergodic. But the converse requires the \nameref{thm:Birkhoff-ergodic}, which is the most important result of ergodic theory.

Most ergodic dynamical systems of interest to probabilists turns out to be strong mixing.

Dyadic transformation

strongly ergodic
completely positive entropy
isomorphic to Bernoulli shift

occurrence time
recurrence time
sojourn time



\begin{namedthm}[Poincaré recurrence theorem]
    $\mu(\{x\in E: T^n x\in E \text{ i.o.}\}) = 1$.
\end{namedthm}

\begin{namedthm}[Von Neumann mean ergodic theorem]
    
\end{namedthm}

\begin{namedthm}[Birkhoff ergodic theorem] \label{thm:Birkhoff-ergodic}
    
\end{namedthm}

induced transformation