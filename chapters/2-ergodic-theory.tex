\chapter{Ergodic theory and stationary processes}
\section{Elementary notions}
Given a probability space $(S,\mathcal S,\mu)$, a \df{measure-preserving transformation} (MPT) $T$ is a measurable function from $(S,\mathcal S)$ to itself such that \[
    \mu(T^{-1}A) = \mu(A)\text{ for all }A\in \mathcal S.
\] In this circumstance we would also say the measure $\mu$ is \df[invariant measure]{$T$-invariant}. The resulting quartet $(S,\mathcal S,\mu,T)$ is called a \df{measure-preserving dynamical system} (MPDS). If $T$ is invertible, and $T^{-1}$ is measurable, then it is equivalent to say $T$ is measure-preserving if \[
    \mu(TA) = \mu(A) \text{ for all }A\in \mathcal S.
\]

We say a measurable function $f$ is \df[(almost) invariant function]{(almost) invariant} (resp.\ strictly invariant) with respect to $T$ is $f \circ T = f$ a.s.\ (resp.\ $x$-pointwise).

Recall $T_*\mu$ is the image measure from \cref{sec:image-measure}. Note $T$ is measure-preserving  precisely means $T_*\mu = \mu$. Therefore by \cref{prop:image-meas-cov}, we have $T$ is measure-preserving if\footnote{follows by just taking $f$ to be any indicator functions} and only if \[
     \int f \,d\mu = \int f\circ T\,d\mu
\] for any $f \in L^+$. This is also true $f \in L^1(\mu)$: breaking $f = f^+ - f^-$, it is clear that $f \circ T \in L^1(\mu)$ is guaranteed. 

An MPT $T$ is said to be $\mu$-\df{ergodic} (or the measure $\mu$ is said to be $T$-\emph{ergodic}) if for all $A\in \mathcal S$, we have \[
    \mu(A \symdiff T^{-1} A) = 0 \implies \mu(A) = 0 \text{ or } 1.
\] A set $A\in \mathcal S$ satisfying $\mu(A \symdiff T^{-1} A) = 0$ is called \df[almost invariant@(almost) invariant]{(almost) invariant}. If instead we have $T^{-1}A = A$, then $T$ is \df{strictly invariant}. The ergodicity of $T$ can be equivalently defined by \[
    T^{-1} A = A \implies \mu(A) = 0 \text{ or } 1,
\] that is, we only need to check strictly invariant sets must be of measure $0$ or $1$.

One direction is obvious. For the other direction, one can check that for any set $A\in \mathcal S$, the set $B = \limsup_n T^{-n} A$ is always going to be strictly invariant.

It is easy to see that the strictly invariant $\sigma$-field $\mathcal I$ and the almost invariant $\sigma$-field must \emph{almost} be the same: \begin{fact}[{\cite[Lemma~25.4]{Kallenberg_2021}}]
    The almost invariant $\sigma$-field is precisely generated by $\mathcal I$ and the $\mu$-null sets in $\mathcal S$.
\end{fact}

\begin{defn}
    An MPDS $(S,\mathcal S,\mu,T)$ is said to be \df[mixing!strong]{strong mixing} if for all $A,B\in \mathcal S$, \begin{equation}
        \lim_n \mu(A\cap T^{-n} B) = \mu(A) \mu(B); \label{eq:strong-mix}
    \end{equation}
    it is said to be \df[mixing!weak]{weak mixing} if for all $A,B\in \mathcal S$,\begin{equation} \label{eq:weak-mix}
        \lim_n \frac{1}{n} \sum_{k=0}^{n-1} \bigl\lvert \mu(A \cap T^{-k} B) - \mu(A) \mu(B) \bigr\rvert = 0,
    \end{equation} i.e., $\bigl\lvert \mu(A \cap T^{-k} B) - \mu(A) \mu(B) \bigr\rvert$ converges to $0$ in the Cesàro sense.
\end{defn}

Hence strong mixing implies weak mixing. In fact weak mixing further implies the system is ergodic. Let $A = B\in \mathcal S$ be strictly invariant, then we may replace $T^{-k} B$ by $B$ in \eqref{eq:weak-mix} and get $\mu(B) = \mu(B)^2.$

Notice that the above argument remains true if we remove the $\abs{\blank}$ in the definition \eqref{eq:weak-mix} of weak mixing. It turns out that \[
    \lim_n \frac{1}{n} \sum_{k=0}^{n-1} \mu(A \cap T^{-k} B) = \mu(A)\mu(B)\quad \text{for all }A,B\in \mathcal S
\] is in fact equivalent to the saying that the system is ergodic. But the converse requires the \nameref{thm:Birkhoff-ergodic}, which is the most important result of ergodic theory.

Most ergodic dynamical systems of interest to probabilists turns out to be strong mixing. Indeed, one may interpret it as eventual independence.

Dyadic transformation

strongly ergodic
completely positive entropy
isomorphic to Bernoulli shift

occurrence time
recurrence time
sojourn time



\begin{namedthm}[Poincaré recurrence theorem]
    $\mu(\{x\in A: T^n x\in A \text{ i.o.}\}) = \mu(A)$.
\end{namedthm}

\begin{proof}
    Consider the set \begin{align*} B \coloneqq \{x\in A : T^n x \notin A \text{ ev.}\} & = \bigcup_{n=1}^\infty \bigcap_{m \geq n} \{x \in A : T^m x \notin A\}, \\
    & = \bigcap_{n=1}^\infty \{x \in A: T^n x \notin A\} \\
    & = A \cap \biggl(\bigcap_{n=1}^\infty T^{-n} (X - A)\biggr).\end{align*} which we want to show is of measure $0$.

    Notice that for any $j < k$ in $\N$, $T^{-j}B$ and $T^{-k}B$ are disjoint because $T^{-j} B \subseteq T^{-k} (X - A)$ while $T^{-k} B \subseteq T^{-k} A$. Therefore \[
        \mu\biggl(\bigcup_{k=1}^\infty T^{-k} B\biggr) = \sum_{k=1}^\infty \mu(B) \leq 1,
    \] and this forces $\mu (B) = 0$.
\end{proof}

\begin{lem}
    Given a $\pi$-system $\mathcal K$ that generates $\mathcal S$, if for each $A \in \mathcal K$ we have $T^{-1} A \in \mathcal S$ and $\mu(T^{-1}A) = \mu(A)$, then $\mu$ is measure-preserving.
\end{lem}

\begin{lem}
    Given a $\pi$-system $\mathcal K$ that generates $\mathcal S$, if \eqref{eq:strong-mix} holds for all $A,B \in \mathcal K$, then the system is strong mixing.
\end{lem}
\begin{proof}
    Apply the $\pi$-$\lambda$ theorem twice
\end{proof}

more general \cite[Lemma~24.2]{Billingsley_1995}

may also use caratheodory as appropriate

\begin{exa}
    \begin{enumerate}
        \item Bernoulli shift or i.i.d.\ sequence strong mixing 
        \nameref{thm:K-01-law}
        \item Rotation on a circle
        \item Markov shift (delayed)
    \end{enumerate}
\end{exa}

from \cite[Theorem~36.5]{Billingsley_1995}
\begin{proof}[Proof of \nameref{thm:HS-01-law}]
    
\end{proof}


\section{The ergodic theorems}

\begin{namedthm}[von Neumann mean ergodic theorem]
    Let $U$ be a contraction operator on a Hilbert space $H$, and let $\Pi$ be the projection onto the closed subspace $\nul(I-U)$. We then have \[
        \frac{1}{n}\sum_{k=0}^{n-1} U^k \to \Pi
    \] in the strong operator topology, i.e., pointwise on $H$.
\end{namedthm}

\begin{proof}

    easy for unitary

    Let $N = \nul(I - U)$, $R = \ran(I - U)$, and $A_n = \frac{1}{n}\sum_{k=0}^{n-1} U^k$.

    If $x \in N$, then $Ux = x$ and $\Pi x = x$, so the convergence holds. If $x \in R$, which means $x = (I - U)v$ for some $v \in H$, then \begin{align*}
        \nm{A_n x} & = \frac{1}{n}\nm{v - U^n v} \\
        & \leq \frac{1}{n}\nm{I - U^n}\nm{v}\\
        & \leq \frac{1}{n}(1 + 1^n) \nm{v} \to 0.
    \end{align*}
    We can extend the convergence above to all $x \in \clos{R}$, essentially because $\nm{A_n} \leq 1$. Take a sequence $\{x_j\}\subseteq R$ converging to $x$. We have for any $n$ and $j$ that \begin{align*}
        \nm{A_n x} & \leq \nm{A_n x_j} + \nm{A_n}\nm{x_j - x} \\
        & \leq \nm{A_n x_j} + \nm{x_j - x}.
    \end{align*} Taking $n \to \infty$ first and $j \to \infty$ next, we have shown $A_n x \to 0$ for all $x \in \clos{R}$.

    The desired claim now follows from the orthogonal decomposition $H = N \oplus \clos{R}$.
\end{proof}

some tricks is needed We follow \cite[Lemma~14.1]{Taylor_2006}.

\begin{namedthm}[Birkhoff pointwise ergodic theorem] \label{thm:Birkhoff-ergodic}
    For $f$ nonnegative measurable or in $L^1(\mu)$, it holds that \[
        \frac{1}{n} \sum_{k=0}^{n-1} f(T^kx) \to \E_\mu(f \giv \mathcal I) \quad\text{a.s.}
    \] If $f \in L^p$, then the convergence also holds in $L^p$.
\end{namedthm}

Consider the context when $T$ is the shift operator on $(S^{\mathbf Z},\otimes^\mathbf{Z} \mathcal S,\mu)$, then for $\mu$-a.e.\ $x \in S^{\mathbf Z}$ that \[
    \frac{1}{n} \sum_{k=0}^{n-1} f(T^kx) \to \hat f(x),
\] where $\hat f$ is $\mathcal I$-measurable and \[
    \int_{G} \hat f\,d\mu = \int_{G} f\,d\mu\quad\text{for all }G\in \mathcal I.
\] Say $X$ is a random variable on $(\Omega,\F,P)$ that is $(S^{\mathbf Z},\otimes^\mathbf{Z} \mathcal S)$-valued with distribution $\mu$. Pushing forward, we get for $P$-a.e.\ $\omega$ that \[
    \frac{1}{n} \sum_{k=0}^{n-1} f\bigl(T^kX(\omega)\bigr) \to \hat f\bigl(X(\omega)\bigr),
\] where $\hat f\circ X$ is $X^{-1} \mathcal I$-measurable, and \[
    \int_A \hat f(X)\,dP = \int_A f(X)\,dP\quad \text{for all }A \in X^{-1}\mathcal I.
\] This confirms our suspicion that when $f$ is nonnegative measurable or $\E \abs{f(X)} \geq 0$, \[
    \frac{1}{n} \sum_{k=0}^{n-1} f(T^kX) \to \E_P[f(X) \giv X^{-1}\mathcal I]\quad\text{a.s.}
\] Clearly $L^p$ convergence should hold as well.

\begin{namedthm}[Maximal ergodic theorem]
    For a $T$-invariant function $g\colon S \to S$ with $g^+ \in L^1(\mu)$, we have \[
        \E\bigl(f - g; f^* - g > 0\bigr) \geq 0.
    \]
\end{namedthm}

\begin{namedthm}[Subadditive ergodic theorem]
    
\end{namedthm}

Consider a one-parameter semigroup $\{T_t\}_{t\geq 0}$ such that \[
    (x, t) \mapsto T_tx 
\] is $\mathcal S \otimes \B[0,\infty)/\mathcal S$-measurable. This is called a \df{measurable flow}. We say the flow is \df[measure-preserving flow]{measure-preserving} if $\mu(T_t^{-1} A) = \mu(A)$ for all $t \geq 0$. Now we naturally define the invariant $\sigma$-field $\mathcal I$ to be the collection \[
    \{I \in \mathcal S : T_t^{-1} I = I\}.
\]

\begin{namedthm}[Continuous-time von Neumann theorem]
    Say there is a one-parameter semigroup $\{U_t\}_{t\geq 0}$ of contraction linear operators on a Hilbert space $H$, and let $\Pi$ be the projection onto the closed subspace $\{x \in H: U_t x = x \text{ for all }t \geq 0\}$ (invariant under $U_t$). We have as the time $N \to \infty$, 
    \[\frac{1}{N}\int_0^N U_t\,dt \to \Pi\] in the strong operator topology, i.e., pointwise on $H$.
\end{namedthm}

\begin{namedthm}[Continuous-time Birkhoff's theorem]
    For $f$ is nonnegative measurable or in $L^1(\mu)$, it holds that as the time $N \to \infty$, we have \[
        \frac{1}{N} \int_0^N f(T_tx)\,dt \to \E_\mu(f \giv \mathcal I) \quad\text{a.s.}
    \] If $f \in L^p$, then the convergence also holds in $L^p$.
\end{namedthm}

Again we should have for $f$ nonnegative measurable or $\E\abs{f(X)} < \infty$, that \[
    \frac{1}{N} \int_0^N f(Q_tX)\,dt \to \E_P[f(X) \giv X^{-1}\mathcal I] \quad\text{a.s.},
\] and also $L^p$ convergence.

\begin{namedthm}[Shannon--McMillan--Breiman theorem]
    Let $H$ be the entropy rate of a given discrete-time finite-state stationary ergodic process $\{X_n\}$, then almost surely \[
        -\frac{1}{n}\log p(X_0,X_1,\dotsc) \to H.
    \]
\end{namedthm}

generalization to countable-state and densities

induced transformation

\section{Invariant measures, ergodicity, and weak convergence}
Throughout this section we may assume $S$ to be a locally compact and separable metric space, and let $T\colon S \to S$ be a measurable mapping.

Occasionally we also assume that $S$ is just compact, so that vague convergence are automatically weak convergence. Recall in this case the space of Borel subprobability measures $\subp(S)$, as the closed unit ball in $C^*(S)$, is a sequentially compact space in the topology of weak convergence. Also $\mathcal P(S)$, as a weakly closed subset of $\subp(S)$ (since mass is preserved), is also a sequentially compact space.

It turns out a nonempty set of invariant measures arises naturally from continuous maps on a compact metric space $S$.

Denote the space of invariant measures by $\mathcal P^T(S)$. It is a closed and convex subset of $\mathcal P(S)$.

\begin{namedthm}[Krylov--Bogoliubov theorem]
    Let $S$ be compact and $T$ be continuous. Given any measure $\nu \in \mathcal P(S)$, we may define a sequence $\mu_n = \frac{1}{n} \sum_{k=0}^{n-1} T_*^k \nu$ of Cesàro sums of image measures.
    
    Any subsequential limit of $\{\mu_n\}$ in the topology of weak convergence is an invariant probability measure. Since $\mathcal P(S)$ is sequentially compact (see \cref{cor:seq-compact-space-prob-meas}), $\mathcal P^T(S)$ must be nonempty.

    (If $S$ is in general locally compact and separable, then if $\{\mu_n\}$ is tight, it has a subsequential limit that is an invariant probability measure, by \cref{prop:tightness-characterization}.)
\end{namedthm}

To make things slightly more general, we may also replace $\nu$ be a sequence of measures $\{\nu_n\} \subseteq \mathcal P(S)$, and define $\mu_n =\frac{1}{n} \sum_{k=0}^{n-1} T_*^k \nu_n$. The same proof below carries over. 
\begin{proof}
    Let $\{\mu_{n_j}\}$ be a subsequence converging weakly to $\mu$. To check $\mu$ is $T$-invariant, it suffices to show that as $j \to \infty$, \[
        \int f\circ T - f\,d\mu_{n_j} \to 0
    \] for all $f \in C(S)$. Expanding the left-hand side, we get \begin{align*}
        & \phantom{{}={}} \frac{1}{n_j} \int \sum_{k=1}^{n_j} f\circ T^{k} - f\circ T^{k-1}\,d\nu \\
        & \leq \frac{1}{n_j} \int \abs{f\circ T^{n_j} - f}\,d\nu \\
        & \leq \frac{2}{n_j} \nm{f}_u \to 0,
    \end{align*} finishing the proof.
\end{proof}

\begin{namedthm}[Continuous Krylov--Bogoliubov theorem]
    Let $S$ be locally compact and separable. Given a measure $\nu$ and a continuous measurable flow $\{T_t\}_{t\geq 0}$, define for each $N > 0$ and any $x \in S$ \[
        \mu_{N,x}(A) = \frac{1}{N}\int_{t=0}^{N} (T_t)_* \nu(A)\,dt.
    \] If the family of probability measures $\{\mu_{N,x}\}_{N > 0}$ is tight for some $x \in S$, then there is an invariant measure $\mu$ with respect to $\{T_t\}_{t\geq 0}$.
\end{namedthm}

     if and only if 

The following result characterizes the ergodic measures among the invariant measures.
\begin{thm}
    Let $T$ be measurable, then the ergodic measures in $\mathcal P^T(S)$ are precisely the extreme points of $\mathcal P^T(S)$.
\end{thm}

\begin{prop}
    Two distinct $T$-invariant measures $\mu$ and $\nu$ must be mutually singular.    
\end{prop}
\begin{proof}
    This is a simple application of the \nameref{thm:Birkhoff-ergodic}. Pick some $B \in \mathcal S$ such that $\mu(B) \neq \nu(B)$. Therefore \begin{equation}
        \frac{1}{n} \sum_{k=0}^{n-1} \ind_B (T^kx) \to \mu(B) \quad \mu\text{-a.s.,} \label{eq:mut-sing-conv}
    \end{equation} and \[
        \frac{1}{n} \sum_{k=0}^{n-1} \ind_B(T^kx) \to \nu(B) \quad \nu\text{-a.s.}
    \] Say \eqref{eq:mut-sing-conv} holds on the set $A$ with $\mu(A) = 1$. It is immediate that $\nu(A) = 0 = \mu(A^\cpl)$, which proves mutual singularity.
\end{proof}




